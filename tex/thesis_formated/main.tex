\documentclass[11pt,a4paper,oneside,listof=totoc,bibliography=totoc]{scrreprt}

\usepackage[utf8]{inputenc}
\usepackage[T1]{fontenc}
\usepackage{csquotes}
\usepackage{lipsum} % for blindtext
\usepackage[backend=biber,url=true,style=numeric,  sorting=none]{biblatex} 
\usepackage{graphicx} % for images
\usepackage{float}
\usepackage{array,booktabs}  %for tables
\usepackage{url}
\usepackage{xspace}
\usepackage{enumitem}

% Configure URL breaking
\usepackage{breakurl}
\Urlmuskip=0mu plus 1mu\relax % Allow URL breaking at any character
\def\UrlBreaks{\do\/\do-\do.\do_\do?\do&\do=\do\%}

\usepackage{hyperref}
\usepackage[ngerman,english]{babel} 
\usepackage{xcolor}
\usepackage{tikz}
\usepackage{amsmath, amssymb}
\usepackage{xargs}
\usetikzlibrary{arrows.meta}
\usepackage{rotating}  % Put this in your preamble
\usetikzlibrary{positioning, shapes.multipart}
\usetikzlibrary{calc}
\usepackage{subcaption}
\usepackage{amsthm}  % already often used for theorems/definitions
\usepackage{listings}
\usepackage{xcolor}
\theoremstyle{definition}
\newtheorem{definition}{Definition}[chapter]

% ----------------------------------------------
% Settings
% ----------------------------------------------
% set bib file (source references.bib)
\bibliography{../master-thesis.bib}
\graphicspath{ {./figures/} }

% Better line breaking and hyphenation
\sloppy % Allow more flexible line breaking
\emergencystretch=10pt % Add extra stretch to help with line breaking
\hyphenpenalty=50 % Reduce penalty for hyphenation
\exhyphenpenalty=50 % Reduce penalty for explicit hyphens
\widowpenalty=10000 % Prevent widows
\clubpenalty=10000 % Prevent orphans

% Better handling of URLs and long text
\usepackage{microtype} % Improves typography and helps with overfull boxes
\microtypesetup{protrusion=true,expansion=true}

%listings
\definecolor{codegray}{gray}{0.95}
\definecolor{keywordcolor}{rgb}{0.36,0.15,0.80}
\definecolor{stringcolor}{rgb}{0.31,0.60,0.02}
\definecolor{commentcolor}{gray}{0.4}

% Global style
\lstset{
  backgroundcolor=\color{codegray},
  basicstyle=\ttfamily\footnotesize,
  keywordstyle=\color{keywordcolor}\bfseries,
  commentstyle=\color{commentcolor}\itshape,
  stringstyle=\color{stringcolor},
  numberstyle=\tiny\color{gray},
  numbers=left,
  stepnumber=1,
  numbersep=10pt,
  tabsize=2,
  showspaces=false,
  showstringspaces=false,
  breaklines=true,
  breakatwhitespace=true,
  captionpos=b
}

\lstdefinelanguage{TypeScript}{
  morekeywords={
    break, case, catch, class, const, continue, debugger, default, delete, do, else,
    enum, export, extends, false, finally, for, function, if, import, in, instanceof,
    new, null, return, super, switch, this, throw, true, try, typeof, var, void,
    while, with, yield, let, await, async, of, as, implements, interface, type,
    from, declare, readonly, keyof, infer, never, unknown, any, boolean, string, number, symbol
  },
  sensitive=true,
  morecomment=[l]{//},
  morecomment=[s]{/*}{*/},
  morestring=[b]",
  morestring=[b]',
  morestring=[b]`,
}

% definitions for tables
\newcommand {\otoprule}{\midrule [\heavyrulewidth]}
\newcolumntype {+}{ >{\global\let\currentrowstyle\relax}}
\newcolumntype {^}{ >{\currentrowstyle }}
 \newcommand {\rowstyle}[1]{\gdef\currentrowstyle{#1} %
 #1\ignorespaces
 }
\newcommand{\tabhead}{\rowstyle{\bfseries}}

\usepackage[colorinlistoftodos,prependcaption,textsize=tiny]{todonotes}
\newcommandx{\unsure}[2][1=]{\todo[linecolor=red,backgroundcolor=red!25,bordercolor=red,#1]{#2}}
\newcommandx{\change}[2][1=]{\todo[linecolor=blue,backgroundcolor=blue!25,bordercolor=blue,#1]{#2}}
\newcommandx{\info}[2][1=]{\todo[linecolor=OliveGreen,backgroundcolor=OliveGreen!25,bordercolor=OliveGreen,#1]{#2}}
\newcommandx{\improvement}[2][1=]{\todo[linecolor=Plum,backgroundcolor=Plum!25,bordercolor=Plum,#1]{#2}}
\newcommandx{\thiswillnotshow}[2][1=]{\todo[disable,#1]{#2}}

% Improved code command that allows line breaks
\usepackage{xparse}
\NewDocumentCommand{\code}{m}{%
  \begingroup
  \ttfamily
  \hyphenchar\font=`\-% Enable hyphenation in typewriter font
  \allowbreak
  #1%
  \endgroup
}

% Alternative inline code command with better breaking
\newcommand{\inlinecode}[1]{%
  \begingroup
  \ttfamily
  \fontdimen2\font=0.4em% Adjust inter-word space
  \fontdimen3\font=0.2em% Adjust inter-word stretch
  \fontdimen4\font=0.1em% Adjust inter-word shrink
  \hyphenchar\font=`\-% Enable hyphenation
  #1%
  \endgroup
}

\usepackage[shortcuts, acronym]{glossaries}
\makeglossaries
\newpage
\section{Acronyms}
\label{sec:acronyms}
\begin{acronym}[AToMPM]
    \acro{glsp}[GLSP]{Graphical Language Server Platform}
    \acro{emf}[EMF]{Eclipse Modeling Framework}
    \acro{mde}[MDE]{Model-Driven Engineering}
    \acro{gui}[GUI]{Graphical User Interface}
    \acro{ide}[IDE]{Integrated Development Environment}
    \acro{sdv}[SDV]{Software-Defined Vehicle}
    \acro{jdt}[JDT]{Java Development Tools}
    \acro{pde}[PDE]{Plug-in Development Environment}
    \acro{sdk}[SDK]{Software Development Kit}
    \acro{api}[API]{Application Programming Interface}
    \acro{uml}[UML]{Unified Modeling Language}
    \acro{xmi}[XMI]{XML Metadata Interchange}
    \acro{xml}[XML]{Extensible Markup Language}
    \acro{lhs}[LHS]{Left-Hand Side}
    \acro{rhs}[RHS]{Right-Hand Side}
    \acro{nac}[NAC]{Negative Application Condition}
    \acro{pac}[PAC]{Positive Application Condition}
    \acro{rpc}[RPC]{Remote Procedure Call}
    \acro{di}[DI]{Dependency Injection}
    \acro{html}[HTML]{Hypertext Markup Language}
    \acro{svg}[SVG]{Scalable Vector Graphics}
    \acro{uri}[URI]{Uniform Resource Identifier}
    \acro{jdk}[JDK]{Java Development Kit}
    \acro{jar}[JAR]{Java Archive}
    \acro{elk}[ELK]{Eclipse Layout Kernel}
    \acro{poc}[POC]{Proof of Concept}
    \acro{uuid}[UUID]{Universally Unique Identifier}
    \acro{atompm}[AToMPM]{A Tool for Multi-Paradigm Modeling}
    \acro{dsml}[DSML]{Domain Specific Modeling Language}
    \acro{vscode}[VS Code]{Visual Studio Code}
  \end{acronym}

\begin{document}
\newcommand{\thesistype}{Master Thesis\xspace}
\newcommand{\studyprogramme}{Software Engineering\xspace}
\newcommand{\department}{Department of Mathematics and Computer Science\xspace}
\newcommand{\group}{Software Engineering Group\xspace}
\newcommand{\student}{Florian Weidner\xspace}
\newcommand{\prof}{Prof. Dr. Gabriele Taentzer\xspace}
\newcommand{\scndProf}{Prof. Dr.-Ing. Christoph-Matthias Bockisch\xspace}
\newcommand{\studentid}{3802094\xspace}
\newcommand{\thesistitle}{Creating web-based diagram editors for specifying and executing model transformations
\xspace}
\newcommand{\submissiondate}{November 07, 2025\xspace}

\begin{titlepage}


\raggedleft

\vspace*{-4cm}

\hspace*{8cm}
\includegraphics{uni-logo.png}

\centering
\Large
\department  
\vspace{0.5cm}\\

\normalsize
\group \\
June 29, 2025

\vspace{1.5cm}
\LARGE
\thesistitle
\vspace{1cm}
\Large
\thesistype

\vspace*{\fill}

\Large
from

\vspace{0.5cm}
\LARGE
\student \vspace{1cm}

\vspace{1cm}

\flushleft
 \Large
\vspace*{\fill}

%-----------
\begin{tabbing}
\normalsize Submission date: \= \normalsize \submissiondate \kill
\normalsize Submission date: \> \normalsize \submissiondate \\
\normalsize First examiner: \> \normalsize \prof \\
\normalsize Second examiner: \> \normalsize \scndProf
\end{tabbing}
%-----------

\end{titlepage}


\pagenumbering{alph}

\input{meta/disclaimer}
\chapter*{Zusammenfassung}
\addcontentsline{toc}{chapter}{Zusammenfassung}

Henshin ist ein leistungsstarkes Modelltransformations-Framework, das auf dem \ac{emf} aufbaut. Trotz seiner Fähigkeiten steht es vor einer Herausforderung bezüglich der Zugänglichkeit für Benutzer: Es erfordert die vollständige Installation der Eclipse IDE, präsentiert eine komplexe Benutzeroberfläche und bietet keine Unterstützung für kollaborative Entwicklung oder cloud-basierten Zugriff. Diese Hürden schränken die Verbreitung ein, insbesondere unter Studierenden, Forschern und verteilten Teams, die von schnellem Experimentieren ohne erheblichen Einrichtungsaufwand profitieren würden.

Diese Arbeit präsentiert \textit{Henshin Web}, eine umfassende webbasierte Modelltransformations-Anwendung, die traditionelle Einstiegshürden beseitigt und gleichzeitig vollständige Kompatibilität mit dem etablierten \ac{emf} Henshin-Ökosystem beibehält. Aufbauend auf der \acf{glsp} und integriert in Eclipse Theia bietet das System browserbasierte grafische Editoren für \ac{emf} Ecore-Metamodelle, Henshin-Transformationsregeln und XMI-Instanzdateien. Die Architektur verwendet ein Client-Server-Design mit TypeScript-basierten Frontend-Komponenten und einem Java-Backend, das das Henshin SDK direkt integriert und damit semantische Äquivalenz mit Desktop-basierten Henshin Transformationen gewährleistet. Die Implementierung adressiert fünf zentrale Forschungsfragen bezüglich der Machbarkeit der Web-Adaption, essentieller funktionaler Anforderungen, Verbesserungen der Zugänglichkeit, Deployment-Strategien und Ökosystem-Integration. Wesentliche technische Beiträge umfassen eine modulare Drei-Editor-Architektur mit spezialisierten Diagramm-Modulen, angepassten Indexierungsmechanismen für \ac{emf}-Modelle unter Verwendung von \acsp{uuid} und Content-Hashes, benutzerdefinierte \ac{ui}-Erweiterungen für Regelauswahl und Parameterspezifikation sowie ein umfassendes Notationsmodell-Management für persistente Layouts. Die Entwicklung erforderte die Konvertierung von Eclipse-Plugins in Maven-Artefakte, die Implementierung mehrerer Indexierungsstrategien für verschiedene Modelltypen und die Erstellung plattformspezifischer Theia-Erweiterungen bei gleichzeitiger Beibehaltung der Kompatibilität über alle möglichen Client-Integrationen hinweg. Die Evaluierung durch Tests validiert die Funktionalität des Systems und demonstriert Verbesserungen der Zugänglichkeit gegenüber traditionellen Eclipse-basierten Workflows. Unit-Tests decken die Kernfunktionalität des Servers ab, während Playwright-basierte End-to-End-Tests vollständige Benutzer-Workflows einschließlich Modell-Laden, grafischer Bearbeitung und Modelltransformations-Anwendung validieren. Die cloud-basierte Bereitstellung über Theia Cloud eliminiert Installationsanforderungen und bietet sofortigen Browser-Zugriff, wodurch die identifizierten Herausforderungen der Benutzererfahrung adressiert werden. Während sich die aktuelle Implementierung auf Kern-Transformationsfähigkeiten konzentriert, bietet die modulare Architektur eine Grundlage für zukünftige Erweiterungen, einschließlich transformation units, State-Space-Analyse, kollaborativer Bearbeitung durch GLSPs Echtzeit-Synchronisationserweiterungen und Integration zusätzlicher MDE-Tools. Das System demonstriert, dass anspruchsvolle Modelltransformationsfähigkeiten durch moderne Webtechnologien bereitgestellt werden können, ohne funktionale Tiefe zu kompromittieren, und etabliert damit eine neue Option für zugängliche modellgetriebene Engineering-Tools.

\makeatletter

\select@language{english}
\makeatother

\chapter*{Abstract}
\addcontentsline{toc}{chapter}{Abstract}

Henshin is a powerful model transformation framework built on the \acf{emf}. Despite its capabilities, it faces an accessibility challenge for users: it requires full Eclipse IDE installation, presents a complex interface, and lacks support for collaborative development or cloud-based access. These barriers significantly limit adoption, particularly among students, researchers, and distributed teams who would benefit from quick experimentation without substantial setup overhead.

This thesis presents \textit{Henshin Web}, a comprehensive web-based model transformation application that eliminates traditional adoption barriers while maintaining complete compatibility with the established \ac{emf} Henshin ecosystem. Built on the \acf{glsp} and integrated into Eclipse Theia, the system provides browser-accessible graphical editors for \ac{emf} Ecore metamodels, Henshin transformation rules, and XMI instance files. The architecture employs a client-server design with TypeScript-based frontend components and a Java backend that directly integrates the Henshin SDK, ensuring semantic equivalence with desktop-based Henshin transformations.

The implementation addresses five core research questions regarding web adaptation feasibility, essential functional requirements, accessibility improvements, deployment strategies, and ecosystem integration. Key technical contributions include a modular three-editor architecture with specialized diagram modules, custom indexing mechanisms for \ac{emf} models using \acsp{uuid} and content hashes, custom \ac{ui} extensions for rule selection and parameter specification, and comprehensive notation model management for persistent layouts. The development required converting Eclipse plugins into Maven artifacts, implementing multiple indexing strategies for different model types, and creating platform-specific Theia extensions while maintaining compatibility across all possible client integrations.

Evaluation through comprehensive testing validates the system's functionality and demonstrates significant accessibility improvements over traditional Eclipse-based workflows. Unit tests cover the core server functionality, while Playwright-based end-to-end tests validate complete user workflows including model loading, graphical editing, and model transformation application. The cloud-based deployment via Theia Cloud eliminates installation requirements and provides immediate browser access, addressing the identified user experience challenges. While the current implementation focuses on core transformation capabilities, the modular architecture provides a foundation for future enhancements including transformation units, state space analysis, collaborative editing through GLSP's real-time synchronization extensions, and integration with additional MDE tools. The system demonstrates that sophisticated model transformation capabilities can be delivered through modern web technologies without compromising functional depth, establishing a new option for accessible model-driven engineering tools.

\makeatletter

\select@language{english}
\makeatother


\tableofcontents

\printglossary[type=\acronymtype, title=List of Acronyms]

% --------------------------------------------
% Main Matter
% --------------------------------------------
\pagenumbering{arabic}

\chapter{Introduction}
\label{sec:introduction}

\section{Background and Motivation}
\label{subsec:motivation}

In software engineering, often \ac{mde} is used to increase development productivity and quality. \cite{transformations-modeldriven} Concepts are modeled closer to the domain, so that they describe important aspects of a solution with human-friendly abstractions. The models can also be used to generate application fragments, that can be directly used as a template source code. In the process of \ac{mde}, many activities need to transform source models into different target models, while following a set of transformation rules. This model transformation process is based on algebraic graph transformations. A metamodel is used to model the structure and rules of the concept. The resulting transformation language can provide automatic model creation, development, and maintenance activities. \cite{transformations-modeldriven} One framework to use \ac{mde} is the \acf{emf} by the Eclipse Foundation. It provides a basis for application development, using modeling and code generation facilities. Many frameworks build upon \ac{emf}, providing various \ac{mde} tools like code generators, graphical diagramming, model transformation, or model validation. \cite{emf} One model transformation framework is Henshin. \cite{henshin-repo} It tries to provide model transformation capabilities with a high level of usability. \cite{henshin-usability} For metamodels it uses \ac{emf} Ecore files. The framework allows to create and apply model transformations on XMI instance files with a defined transformation language. It provides a graphical and textual syntax to create these transformation rules. \cite{henshin-repo} Henshin can be used as a Eclipse plugin. For new users, Eclipse \ac{ide} needs to be installed and the heavy editor makes the use of Henshin unintuitive without prior experience.
Therefore, the goal exists to create a graphical editor to use the Henshin model transformations without the overhead of the heavy \ac{ide}, that has to be installed. A web-based graphical editor would make the use of Henshin even more accessible and intuitive.

\ac{glsp} is an open-source framework by the Eclipse Foundation, which can be used to build a web-based Henshin graph editor. The framework is used to develop custom diagram editors for distributed web-applications. \cite{glsp-repo} It can provide graph editors for the Eclipse Desktop \ac{ide}, Eclipse Theia, \ac{vscode} and a standalone version usable in any website. It brings the support of \ac{emf} models as a data source and the functionality of the existing Henshin \ac{sdk} can be called from the Java server of \ac{glsp}. \cite{glsp-doc} With these functionalities, \ac{glsp} fits to create an easy accessible, intuitive application to create and apply Henshin model transformations called \textit{Henshin Web}. You can find the repository of \textit{Henshin Web} here: \url{https://gitlab.uni-marburg.de/weidnerf/henshin-web-model-transformation}.

\section{Problem Statement}
\label{subsec:problem-statement}

While Henshin offers powerful model transformation capabilities, its integration exclusively as an Eclipse \ac{ide} plugin creates barriers in accessibility for potential users. Users that want to work with Henshin must first install and properly configure the entire Eclipse \ac{ide} environment, a prerequisite that narrows the framework's practical accessibility and constrains the frameworks reach and usability.

The Eclipse installation requirement itself acts as a notable entry barrier. Students and researchers exploring \ac{mde} concepts often want to experiment with transformation rules without investing time in setting up a comprehensive development environment. Yet Henshin's current distribution model forces that. There are complicated installation procedures, environment configurations to overcome and the need to familiarize oneself with Eclipse's interface—all before actually engaging with Henshin's core functionality.

Eclipse as a platform introduces its own complications. As a feature-rich development environment designed for professional software development, Eclipse's extensive capabilities and complex interface can overwhelm users whose objective is creating and executing model transformations. Accomplishing transformation tasks requires navigating through numerous wizards, views, and menu structures, which slows down work and creates friction in the workflow.
The default tree-based editors for Ecore metamodels, Henshin rules, and \ac{xmi} instances become difficult to work with as models grow in size and complexity. For graphical editing of transformation rules, a separate initialization of diagram files has to be manually executed. Similarly, visualizing Ecore metamodels graphically requires specific diagram setup steps. Working with \ac{xmi} instance files demands additional extension installations. Most notably, applying transformation rules lacks any graphical support. Users have to start a wizard to be able to apply model transformations on a \ac{xmi} instance.

Collaboration and flexibility suffer under the current environment. Teams cannot easily share model transformation examples or work together on the same rules within Eclipse. The use of \ac{vcs} brings some colaborative elements, but real-time collaboration or user based workspace access across multiple devices remains impossible.

These accessibility and usability challenges lead to inefficienct workflows. Newcomers encounter steep learning curves before they can productively use the tool, while experienced users who need straightforward access to model transformation features must tolerate unnecessary and unused \ac{ide} complexity.

\section{Research Questions}
\label{subsec:research-questions}

Based on the identified problems with the current Eclipse-based approach to Henshin model transformations, this thesis aims to address the following research questions that guide the development and evaluation of a web-based solution:

\begin{researchquestion}
\label{rq:web-adaptation}
    How can Henshin model transformation capabilities be effectively adapted for web-based environments?
\end{researchquestion}

This question investigates if translating the desktop-based Henshin functionality into a web application is techincally possible and what architecture can fulfill all requirements. It examines how the core model transformation engine, metamodel handling, and rule definition capabilities can be preserved while adapting to web technologies and browser constraints.

\begin{researchquestion}
\label{rq:functional-requirements}
    What are the essential functional requirements for a web-based Henshin editor that maintains usability while reducing complexity?
\end{researchquestion}

This question focuses to identify the minimum viable feature set that is needed to provide meaningful transformation capabilities in order to create an application that can completely handle typical use cases.

\begin{researchquestion}
\label{rq:accessibility-ux}
    How does a web-based approach improve accessibility and user experience compared to the traditional Eclipse plugin?
\end{researchquestion}

This question evaluates the accessibility and usability of the web-based solution. It examines metrics such as installation complexity, learning curve, collaboration capabilities, and overall user satisfaction when working with model transformations.

\begin{researchquestion}
\label{rq:deployment-strategies}
    How do different deployment strategies affect the accessibility, usability, and adoption barriers for web-based model transformation tools?
\end{researchquestion}

This question explores various deployment options for the \textit{Henshin Web} editor. That includes a standalone web applications, cloud-hosted services, or distribution through a desktop application. It assesses how these strategies impact user access, usability, and the overall adoption of the tool among different user groups.

\begin{researchquestion}
\label{rq:ecosystem-integration}
    How can the web-based editor integrate with existing \ac{emf} and Henshin ecosystems?
\end{researchquestion}

This question explores the compatibility and interoperability requirements. It ensures that the web-based solution can work with existing metamodels, transformation rules, and instance files created in the traditional Eclipse environment.

These research questions address the goal of creating an accessible, intuitive, and fully functionally  web-based alternative to the current Eclipse-dependent Henshin workflow.

\section{Scope and Limitations}
\label{subsec:scope-limitations}

This thesis focuses on developing a web-based solution for Henshin model transformations. It contains specific boundaries and constraints that define the research scope. The primary scope includes the design, implementation, and evaluation of a web-based editor using the \ac{glsp} framework. The application should provide the core Henshin model transformation functionality. The work includes adapting the essential features of the Henshin Eclipse \ac{ide} plugin into the web environment. The development focuses on transformation rule creation, metamodel handling, and instance file processing. The implementation should include the workflow of loading \ac{emf} Ecore metamodels, creating transformation rules through a graphical interface, and applying these transformations to \ac{xmi} instance files.

The research addresses accessibility improvements to minimize the initial challenge for beginners, where users need quick access to model transformation capabilities. There they don't need an extensive setup and can directly work with the model graphically. The evaluation covers usability aspects, and functional completeness to cover most meaningful use cases. The system should integrate with the existing \ac{emf} and Henshin ecosystems, to ensure compatibility with established workflows and file formats.

There are some limitations that constrain the scope of this research. On is that the web-based implementation does not aim to replicate every advanced feature from the Eclipse Henshin plugin. Complex transformation scenarios, advanced debugging capabilities are beyond the current scope. The focus remains on core functionality that serves the primary use cases that are defined in the requirements analysis.

The evaluation only covers a limited set of scenarios and user interactions. While the research aims to demonstrate improvements over the Eclipse approach, comprehensive studies or extensive industrial validation are not part of the scope of this thesis work.

Additionally, the research does not extend to develop new transformation algorithms or enhancing the existing Henshin transformation engine. The focus remains on bringing Henshin into the web environment achieving high accessibility and usability, rather than advancing the theoretical foundations of model transformation techniques.

A system constraint is that the backend has to be Java-based, to be able to directly run the Henshin \ac{sdk}. 

These scope definitions and limitations ensure that the research remains focused and achievable within the constraints of a master's thesis while addressing the core problems, that were identified.

\section{Structure of the Thesis}
\label{subsec:structure-thesis}

In this thesis each chapter is building upon the previous ones to provide a full view of the development and evaluation of the \textit{Henshin Web} application. The rest of the thesis is structured as follows:\\

Chapter \ref{sec:background} introduces the basic technologies and concepts that are used to build the application. It covers the Eclipse Foundation ecosystem, \ac{emf} as the modeling framework, Henshin for model transformations, and \ac{glsp} as the web-based graphical editing platform.\\
Chapter \ref{sec:related-work} looks at different model transformation tools and web-based modeling solutions. It presents scientific literature on model transformation software, analyzes existing tools and their limitations, and compares various web-based modeling environments.\\
Chapter \ref{sec:requirements} defines the functional and non-functional requirements for the Henshin Web editor. It identifies potential user groups, defines the system scope and context, and details the specific capabilities the application must provide.\\
Chapter \ref{sec:system-design} presents the overall system architecture of Henshin Web. It describes important design decisions, different component interactions, and describes architectural patterns. \\
Chapter \ref{sec:implementation} shows the concrete implementation of core components within the Henshin Web application. It covers the technical realization of key features, integration challenges, and solutions developed to bring Henshin into the web. \\
Chapter \ref{sec:testing} discusses the testing strategy employed to validate the application's functionality. It describes the unit testing approach, the end-to-end testing environment, and presets test cases to check the application's behavior.\\
Chapter \ref{sec:deployment} compares various deployment strategies and options for making Henshin Web accessible to users. It examines different hosting approaches, infrastructure requirements, and considerations for scalability and maintenance.\\
Chapter \ref{chap:usage} provides a user guide how to use Henshin Web. It includes guides for creating and editing metamodels, transformation rules, and instance files. It also contains an administrative guide for user management and system configuration. This chapter serves as practical documentation for both end users and system administrators.\\
Chapter \ref{sec:conclusion} concludes the research findings, evaluates the success of the approach in addressing the identified problems, and reflects the success of web-based model transformation tools. It discusses limitations of the current implementation, potential future enhancements.
\input{chapters/theoretical_background.tex}
  \chapter{Related Work}
  \label{sec:related-work}

  \section{Scientific Literature}
  \label{subsec:related-scientific-literature}

  This section reviews relevant scientific literature that directly relates to the development of web-based modeling tools using \ac{glsp} and the migration of Eclipse-based modeling frameworks to web environments. The selected works provide practical insights into building \ac{glsp}-based applications, migrating \ac{mde} tools from Eclipse to web technologies, and improving the usability of model transformation frameworks like Henshin.

  \subsection{GLSP-Based Web Modeling Tools}

  \citeauthor{bork2023vision} present in \citeyear{bork2023vision} a comprehensive vision for flexible web-based modeling tools built with \ac{glsp} \cite{bork2023vision}. Their work addresses the same challenge that Henshin Web faces: creating modern diagram editors that work seamlessly across different platforms including web browsers, \ac{vscode}, Eclipse Theia, and the Eclipse desktop \ac{ide}. The authors describe how \ac{glsp}'s client-server architecture enables modeling tools to be platform-independent while maintaining professional-grade functionality. They emphasize that web-based modeling tools must provide the same quality as traditional desktop applications while leveraging web-specific advantages such as easier deployment, platform independence, and collaboration features. 
  
  The paper discusses architectural patterns for \ac{glsp}-based tools, including how to structure the server-side model management and client-side rendering. This directly informs the architecture of Henshin Web. The authors also address extensibility concerns, showing how \ac{glsp} applications can be customized through custom actions, tool palette items, and context menu contributions. These are features that Henshin Web utilizes for editing and applying transformation rules. Furthermore, the paper discusses integration possibilities with other Eclipse technologies, particularly \ac{emf}. This is crucial for Henshin Web's need to work with Ecore metamodels and \ac{xmi} instance files. The work validates the choice of \ac{glsp} as an appropriate framework for bringing Henshin to the web while maintaining compatibility with the existing Eclipse-based ecosystem.

  \subsection{Practical Experience with GLSP Development}

  \citeauthor{metin2023glsp} provide in \citeyear{metin2023glsp} valuable practical insights through their experience developing bigUML, a web-based \ac{uml} editor built with \ac{glsp} \cite{metin2023glsp}. Their paper reports on real-world challenges and solutions when building a production-ready \ac{glsp}-based modeling tool, making it highly relevant for the development of Henshin Web. The authors identify key architectural decisions, such as how to organize the server-side model state, handle concurrent editing scenarios, and implement complex editing operations that span multiple model elements. 
  
  They discuss performance considerations, particularly regarding the efficiency of transforming models into graphical representations when working with large diagrams. This concern is equally relevant when displaying complex Henshin transformation rules or large instance models. The paper also addresses the learning curve for developers new to \ac{glsp}, documenting common pitfalls and best practices that can speed up development. For example, they describe strategies for implementing custom validation logic, managing undo/redo operations, and integrating with external services. Their experience with deployment and operation of web-based modeling tools provides insights into hosting strategies, scalability considerations, and user access management. 
  
  The authors also discuss the benefits they observed in practice, including easier onboarding for new users who can access the tool immediately through a web browser without installation, and improved collaboration capabilities. These lessons learned directly inform the development approach for Henshin Web, helping to avoid common mistakes and adopt proven patterns for \ac{glsp}-based tool development.

  \subsection{Migration from Eclipse to Web Technologies}

  \citeauthor{domros2018moving} describes in \citeyear{domros2018moving} the migration of the KIELER modeling tool from the Eclipse desktop platform to web technologies using the Theia framework \cite{domros2018moving}. This work is particularly relevant as it addresses the same challenge that Henshin Web faces: adapting an existing Eclipse-based modeling tool for web environments. The thesis documents a systematic approach to migration, identifying which components can be reused, which require adaptation, and which need complete reimplementation. KIELER, like Henshin, is an Eclipse plugin that provides graphical modeling capabilities, making the migration challenges comparable. 
  
  Notably, while this thesis predates \ac{glsp}, KIELER uses Sprotty for diagram rendering—the same underlying \ac{svg}-based diagramming framework that \ac{glsp} is built upon. Later versions of KIELER integrated with \ac{glsp}, demonstrating the evolution path from Sprotty-based custom solutions to \ac{glsp}-based standardized approaches. The author analyzes the architectural differences between Eclipse and Theia, discussing how Eclipse's plugin system and extension points map to Theia's modular architecture. 
  
  Key findings include the importance of separating business logic from \ac{ui} concerns to enable cross-platform compatibility, and strategies for using the Language Server Protocol to share code between different frontend platforms. The thesis also evaluates the user experience improvements achieved through the web-based approach, including faster startup times, reduced installation complexity, and improved accessibility. 
  
  The author identifies challenges such as adapting keyboard shortcuts that conflict with browser behaviors, handling file system access in a sandboxed web environment, and managing the increased complexity of a client-server architecture. These insights provide valuable guidance for the Henshin Web migration strategy, particularly regarding which Eclipse-specific features to preserve and how to implement them in a web context.

  \subsection{Henshin Framework and Usability}

  \citeauthor{struber2017henshin} present in \citeyear{struber2017henshin} the current state of the Henshin framework, emphasizing its focus on usability for model transformation development \cite{struber2017henshin}. This work is essential context for Henshin Web as it documents the features and design philosophy of the framework being adapted for web environments. The authors describe how Henshin provides both graphical and textual syntax for defining transformation rules, supporting different user preferences and use cases. They discuss the algebraic graph transformation foundations that ensure formal correctness and verifiability of transformations. Henshin Web must preserve these foundations in its web-based implementation. 
  
  The paper presents Henshin's current tooling, including the Eclipse-based graphical editor for transformation rules, the tree-based editor for transformation units, and the interpreter for executing transformations. Understanding these existing tools is crucial for designing their web-based counterparts. The authors emphasize Henshin's integration with \ac{emf}, showing how transformation rules are typed over Ecore metamodels and how transformations operate on \ac{xmi} instance files. These are core concepts that Henshin Web must maintain. 
  
  The paper also discusses usability features such as visual differentiation between preserve, create, and delete actions through stereotypes, support for \acp{nac} and \acp{pac}, and parameterized rules. These features represent the baseline functionality that Henshin Web should provide to ensure users can perform the same transformation tasks as in the Eclipse version. The paper's focus on usability aligns directly with Henshin Web's goal of making model transformations more accessible by reducing the barrier to entry through a web-based interface.

  \section{Existing Tools and Technologies}
  \label{subsec:related-tools}

    There are many existing tools for model transformations. \citeauthor{kahani2019survey} created a survey in \citeyear{kahani2019survey} of various model transformation tools. They classified 60 different tools, including Henshin. In Figure \ref{fig:tools-environments}, you can see how many tools provide specific execution environments. 73\% of the tools provide plugins for the Eclipse \ac{ide}, and 20\% of the tools are integrated or dependent on other \acsp{ide}. 18\% have no \ac{ide} support, and only two tools are web-based. In total, 89\% of the tools have external dependencies such as an \ac{ide} or other tools. Dependencies often complicate the installation and usage of the tool. \cite{kahani2019survey}

  \begin{figure}[h]
    \centering
    \includegraphics[width=0.6\textwidth]{model-tools.png}
    \caption{Execution environments of model transformation tools. Image obtained from \cite{kahani2019survey}}
    \label{fig:tools-environments}
  \end{figure}

  One web-based tool included in the survey is \ac{atompm} \cite{atompm}. It is a web-based modeling tool to create \ac{dsml} environments, performing model transformations and manipulating and managing models. \cite{atompm} It was created in \citeyear{atompm} and supports all model transformations that are based on T-Core \cite{tcore}, a minimal common basis that allows interoperability between different model transformation languages. \cite{tcore} Metamodels can be defined with a simplified \ac{uml} language. The graphical modeling environment offers debugging and the ability to collaborate and share modeling artifacts in the browser. \cite{atompm}


  There are also other web-based tools for \ac{mde}. WebGME \cite{webGME} is a web-based modeling tool, created in \citeyear{webGME}. It allows to collaboratively design \acp{dsml} using model versioning and broadcasting changes to all active users. It supports prototypical inheritance, where any model can be instantiated recursively, so changes are propagated down the inheritance tree. It also provides scalability, collaborative modeling and model versioning. Metamodels and compositions can be created with WebGME, but no graph transformations can be applied to a model. Even though model transformations are not possible, the editor was one of  the first solutions for web-based modeling tools. \cite{webGME} The software provides extension points to customize or extend the software, but no model transformation capabilities were added by any available extension. \cite{webgme-website} The tool is still hosted and maintained, to be used for free. \cite{webgme-website}


  WebDPF \cite{webDPF} is another web-based modeling tool, published in \citeyear{webDPF}. Compared to WebGME and \ac{atompm}, it supports model navigation and element filter capabilities, a JavaScript editor for writing predicate semantics, reusability of transformation rules, partial model completion, and a termination analysis. These features try to improve the usability of the tool. \cite{webDPF} Even though the tool had improvments upon existing tools, the originally mentioned hosted WebDPF portal is offline by now. 


  There is also a \ac{glsp}-based Ecore metamodel editor, created by the \ac{glsp} development team. It was implemented with the \ac{glsp} version 0.9 but never updated further. It allows to create and edit \ac{emf} Ecore models in a Theia web editor. Even though the project cannot be used directly, due to the use of another source model format and breaking changes in major updates of the \ac{glsp} framework, it provides various classes that can be used as a template for the Henshin Web Ecore viewer. One example is the factory code that maps the \ac{emf} Ecore model to the graphical model. \cite{glsp-ecore-repo}
  The findings show, that there are many existing model transformation tools, but only very few web-based solutions, that provide an easy entry into \ac{mde} and model transformations. Henshin web tries to fill this gap.
  \chapter{Requirements Analysis}
  \label{subsec:requirements}

  The purpose of this chapter is to systematically identify, analyze, and document the requirements of the software system developed in the context of this thesis. The chapter outlines functional and non-functional requirements as well as the system stakeholders and constraints.

\section{Stakeholders and User Needs}
  \label{subsec:stakeholders}

  \begin{itemize}
    \item \textbf{Students}: Students who want to learn about \ac{mde} and transformation rules, want a very simple and intuitive entry into the topic. Trying out transformations in the browser is a good start, without having to install a lot of software. For Students the core functionality is sufficient, as they only want to try out transformation rules and learn how they work.
    \item \textbf{Researchers}: Researchers, that are researching for \ac{mde} and come arcross Henshin, want to be able to test or try out model transformations of Henshin. For Researchers the core functionality could be sufficient, but editing capabilites of the transformation rules and metamodels are pracitcal for them.
    \item \textbf{Software Engineers}: Software engineers, that are using \ac{mde}, want to be able to test their transformation rules. They want a powerful editor and a colaborative environment to work on their models. For Software Engineers the defined additional functionality as well as some code generation support are needed to cover their needs.
  \end{itemize}

  The different Stakeholders show, that for more features the application provides, more users can be reached. With more features and use cases the application can cover, it provides more value for enterprise users, that work for production systems.

  \section{System Scope and Context}
  \label{subsec:system-scope}

  The Eclipse \acs{ide} plugin of Henshin works as a template for the functionality of the application. It provides functionality to create, edit and apply Henshin transformation rules for Ecore metamodels on \ac{xmi} instances. To create a full enterprise application, that will get used for projects in the industry, the application has to also provide very similar functionality.
  The defined core functionality is the minimum set of features that the application must provide to be useful for the users. They are only a subset of requirements to provide a full web-based copy of the Henshin Eclipse plugin.

  The core functional requirements extends already existing functionality, that Eclipse Theia and \ac{glsp} already provide. Theia provides various views like the explorer for basic file management, including opening, saving, closing, creating and deleting files, a problems view, a integrated terminal, or edit operations like copy or paste. \ac{glsp} provides default functionality for each graphical editor. That includes selection of elements, moving nodes, realinging edges, zooming, or moving and reseting the viewport. Most of these features can be further configured to be able to create an editor fitting the specific needs.

  \section{Functional Requirements}
  \label{subsec:functional-requirements}

  The main use case that the application should support is that a user can try out transformation rules on \ac{emf} \ac{xmi} instance files. In this usecase, the user already has a metamodel and transformation rules. He wants to test the transformation rules on various instances in an accessible, intuitive, and easy to use graphical editor. From that use case the following core functional requirements can be derived.

  \textbf{Core Functionallity:}
  \begin{itemize}
    \item \textbf{R1:} \ac{emf} \ac{xmi} instance files should be displayed in a graphical editor. That contains the nodes, edges and attributes of the model.
    \item \textbf{R2:} The \ac{xmi} instance editor should provide editing functionality to create, update and delete nodes, edges and attributes.
    \item \textbf{R3:} In the \ac{xmi} instance editor all applicable transformation rules should be listed.  When a rule is selected to get applied, all parameters have to be specifiable.
    \item \textbf{R4:} When a rule gets appplied, the graphical editor of the \ac{xmi} instance should be updated to reflect the changes made by the transformation rule. The application should also support undo and redo functionality for the applied transformation rules.
    \item \textbf{R5:} Henshin transformation rule files should be displayed in a graphical editor. That contains the nodes, edges and attributes of the model and their action types. The user should be able to switch between all rules of a \textit{.henshin} file.
    \item \textbf{R6:} \ac{emf} Ecore metamodel should be displayed in a graphical editor. That contains the nodes, edges and attributes of the metamodel.
\end{itemize}

Next to the core use case, the second use case is that a user wants to create a full transformation language from scratch, that can be used to model and test the system and generate production code from it. In this use case, the user wants to create and edit metamodels and transformation rules. To support this use case, the following additional functional requirements are defined:

\textbf{Additional Functionallity:}
  \begin{itemize}
    \item \textbf{R7:} The Henshin rule editor should provide editing functionality to create, update and delete nodes, edges, attributes and their action types.
    \item \textbf{R8:} The Ecore metamodel editor should provide editing functionality to create, update and delete nodes, edges and attributes.
    \item \textbf{R9:} Henshin transformation units are also listed in the \ac{xmi} instance editor and can be applied.
    \item \textbf{R10:} Show the possible transformation rule matches in the \ac{xmi} instance editor, when selecting a transformation rule.
    \item \textbf{R11:} Provide the functionality to apply a State Space analysis on a \ac{xmi} instance.
    \item \textbf{R12:} Provide the functionality to apply a conflict and dependency analysis on a \ac{xmi} instance. 
  \end{itemize}

  There exist many more use cases for model transformations and \ac{mde} in general. The application can grow to a web-based platform for \ac{mde} in th future. Additional functionality will be disscused in section \ref{subsec:suggestions-future-development} but these usecases are not scope of this thesis.

  \section{Non-Functional Requirements}
  \label{subsec:non-functional-requirements}

  In addition to the core functionality, the system must meet several non-functional requirements:

  \textbf{Non-Functional Requirements:}
  \begin{itemize}
    \item \textbf{R13:} The application should be web-based and preferably accessible via a web browser.
    \item \textbf{R14:} The application should be responsive and work on different screen sizes. It does not have to support mobile devices and touch interactions, since \ac{glsp} is also not supporting touch interactions \cite{glsp-repo}.
    \item \textbf{R15:} The application should be user-friendly and intuitive to use. For that, the application should follow the design principles of \ac{glsp} and Eclipse Theia. That includes the use of views of theia, like the exploerer and the predefined UI contorls of \ac{glsp}, like the tool palette or the context menu.
  \end{itemize}

  \section{System Constraints}
  \label{subsec:system-constraints}

  One constraint is the use of Henshin as a tranformation language. Henshin is a Java-based framework, which menas that the application needs a possibility to run Java code in the backend. The easiest way for that is to use a Java-based backend, that can directly use the Henshin SDK code.
  The use of Henshin also brings the constraint that, metamodels and instances are based on \ac{emf}.

  Another constraint is the use of web-based technologies and preferably a resulting web application. For model transformations, there exist many applications, but not many of them are web-based. This constraint is also a non functional requirement and was also motivated in previous sections.
  The initial version of the application will support English only.
  \section{System Design and Architecture}
  \label{sec:system-design}

    \ac{glsp} uses a client-server architecture. The client and server communicate via a websocket connection and JSON-RCP. The backend will be implemented in Java, which is the main programming language for \ac{emf} and especially the Henshin SDK. The \ac{glsp} backend supports integration with \ac{emf} models as the underlying source model for the diagrams. That makes the integration of Henshin easier because all files needed are based on \ac{emf}. The \textit{HenshinRessourceSet} can be loaded directly over the \ac{emf} integration of \ac{glsp} into the \textit{ModelState}. For the XMI instance files, the Henshin rule files, and the Ecore metamodel files, a mapping to the \ac{glsp} internal graphical model needs to be implemented. \cite{eclipseGLSP}

  For providing editing functionality, all commands need to be implemented as Actions and Handlers in the \ac{glsp} backend. They are manipulating the source files directly. The \ac{glsp} architecture is built that after manipulating the source model, the new state is mapped to the graphical model of \ac{glsp} again and then displayed in the client. \cite{eclipseGLSP} 

  The \ac{glsp} client is implemented in TypeScript. It uses Sprotty, an SVG-based diagramming framework, to render the diagrams.
  Different diagram languages can be defined for multible file types. For each file type, a different backend module must be registered. The ecore metamodel, the Henshin rule file and the XMI instance files should be in the same folder in a workspace. There you can open the corresponding file and EMF model should be displayed in a graphical editor.

  \subsection{High-Level Architecture}
  \label{subsec:high-level-architecture}

  \subsection{Component Design}
  \label{subsec:component-design}

  \subsection{Data Flow and Control Flow}
  \label{subsec:data-flow}

  \subsection{Data Models and Structures}
  \label{subsec:data-models}

  \subsection{User Interface Design}
  \label{subsec:user-interface-design}
\chapter{Implementation}
  \label{sec:implementation}
  This chapter describes the development process and shows the solution and implementation of specific problems, that appeared while implementing the application. The full implementation of \textit{Henshin Web} can be found here: \url{https://gitlab.uni-marburg.de/weidnerf/henshin-web-model-transformation}. The first challenge was to integrate the Henshin \ac{sdk} into the \ac{glsp} project. Another challenge was to index the elements of the different \ac{emf} models and formats. One big \ac{ui} decision was where to place the selection of transformation rules in the application.

  \section{Development Process}
  \label{subsec:development-process}
  The development of the \textit{Henshin Web} \ac{glsp} editor was done by one person in a time span of about 6 months. Derived from the functional requirements (see section \ref{subsec:functional-requirements}) the project was split into 7 milestones. The milestones were defined as follows:
  \begin{itemize}
    \item \textbf{Milestone 1:} Setup the project and create a diagram editor that can display \textit{.xmi} files.
    \item \textbf{Milestone 2:} Create editing capabilities for \ac{xmi} instance files.
    \item \textbf{Milestone 3:} Henshin transformation rules can be displayed and applied to the instance model.
    \item \textbf{Milestone 4:} Create an additional diagram editor that can display Henshin rules.
    \item \textbf{Milestone 5:} Create an additional diagram editor that can display Ecore metamodels.
    \item \textbf{Milestone 6:} Create editing capabilities for Henshin rules.
    \item \textbf{Milestone 7:} Create editing capabilities for Ecore metamodels.
  \end{itemize}

  Each milestone was split into smaller issues. The first milestone was used to create a \ac{poc} to test the integration of Henshin into a \ac{glsp} project. In this phase, the focus was to get to know how the frameworks \ac{emf}, \ac{glsp}, Henshin and Eclipse Theia work. Even though \ac{glsp} provides a well structured documentation and project templates, they didn't cover many use cases for the development of Henshin Web. Henshin also doesn't provide documentation of their \ac{api}. Therefore for these frameworks a lot of source code reading and understanding was needed. 
  Git was used as a version control system. The development of a milestone was done in a separate development branch. When all features of the milestone were implemented, the state of the application was additionally tested and then merged into the main branch.

  \section{Tooling and Environment}
  \label{subsec:tooling}
  For the development of Henshin Web, \ac{vscode} \cite{vscode} was used to develop the client and IntelliJ IDEA \cite{intellij} was used to develop the Java server.
  For understanding the source code of not well documented frameworks, the use of Chatbot Agents was very helpful. I used Github Copilot in \ac{vscode} with the model Claude Sonnet 4 \cite{claude_sonnet} in agent mode. Because it has access to the source code of the dependent frameworks, it can search for specific classes or methods or explain certain concepts.
  Git was used as version control system and GitLab was used as a remote repository. It was also used for the project management, where the milestones were defined and the issues were created. A issue board was used to show the current state and progress of the project. The GitLab package registry was also used to store the Henshin maven packages, to be able to access them from the \ac{glsp} project. These packages are then available to every contributor of the GitLab project, who wants to develop on the project. More about the creation of Maven packages will be shown in the next section.

  \section{Code Examples}
  \label{subsec:code-examples}

  This section shows the solution and implementation of specific problems, that occurred during the development process.

  \subsection{Integration of Henshin into a GLSP project} 
  \label{subsec:henshin-glsp}

  The Henshin source code provides both the Eclipse \ac{ide} plugin and a Java SDK for using the Henshin interpreter. The project of Henshin is structured as an Eclipse project and is available as a set of Eclipse plugins and features. \cite{henshin-repo} On the other hand, \ac{glsp} projects typically use a Maven project structure. \cite{glsp-repo} To add dependencies to a Maven project, the dependencies should ideally be available as Maven artifacts. However, Henshin doesn't provide a Maven artifact, since that is not needed for an Eclipse plugin. The Henshin version 1.8.0 is compatible with \ac{jdk} 11 and higher. \ac{glsp} version 2.3.0 has the prerequisite of \ac{jdk} 17. Therefore, the versions are compatible to run together. The Henshin code consists of 45 plugins, of which 22 are contained in the Henshin SDK, that we need as a dependency in our \textit{Henshin Web} \ac{glsp} project. Each plugin can be downloaded as a \ac{jar} file. To create Maven packages from the \acsp{jar}, a PowerShell script is used. It reads all \acsp{jar} files from a folder, renames them to the correct Maven artifact name, creates a basic \code{pom.xml} file for them, deploys them to the GitLab package repository, and creates a list that needs to be included in the Maven \code{pom.xml} file of the \ac{glsp} project. A package of each plugin is created, because for the \textit{Henshin Web} editor, only some parts of the Henshin SDK are needed. To use the Henshin model package, the additional dependency of the Nashorn JavaScript engine \cite{nashorn-repo} is needed. The Nashorn engine is used to execute calculation expressions of transformation rules. \cite{henshin}

  \subsection{GModelFactory}
  \label{subsec:gmodel-factory}

  The heart of a \ac{glsp} server diagram module is the \code{GModelFactory}. It is responsible for creating the graphical model from the source model. In listing \ref{lst:gmodel-factory} you can see the implementation of parts of the creation of the graphical nodes. The method \code{fillRootElement(GmodelRoot newRoot)} gets called when a new graphical model should be created. It fetches the source model elements from the \code{ModelState}, iterates over them, and creates \code{GNode} elements using a builder pattern. In the method \code{createNode(DynamicEObjectImpl eObject)}, it is configured how the node should look in the editor. It sets the id, adds \ac{css} classes and configures rounded corners. It also sets the type of the node. The type can be a default type, or custom types, that have their own customized client implementation. In listing \ref{lst:gmodel-factory} you can see for example that the root node of the \ac{xmi} instance model gets a different type. The type is used to configure that only one root node can exist and that it cannot be deleted, if other child nodes exist. The method \code{applyShapeData(eObject)} adds the layout information from the notation model. In the \code{GNodeBuilder} also the built child elements like the header or the attributes are added. This creates a tree structure of graphical elements, that are all attached to the \code{GModelRoot}. The \code{RuleGModelFactory} and the \code{EcoreGModelFactory} work similar to the \code{XMIGModelFactory}, but they create different node types and handle the source model elements differently.

  \begin{lstlisting}[language=Java, caption={Parts of \code{XMIGModelFactory}}, label={lst:gmodel-factory}]
@Override
protected void fillRootElement(GModelRoot newRoot) {
    EGraph instanceNodes = modelState.getInstanceGraph();

    newRoot.getChildren().addAll(instanceNodes.stream()
            .map(eObject -> (DynamicEObjectImpl) eObject) //
            .map(this::createNode)//
            .toList());

    ...
}

public GNode createNode(DynamicEObjectImpl eObject) {
    String type = DefaultTypes.NODE;
    if(eObject.eContainer() == null) {
        type = HenshinTypes.XMI_ROOT_NODE;
    }

    GNodeBuilder b = new GNodeBuilder(type) //
            .id(UUIDAdapter.getOrAssignId(eObject)) //
            .layout(GConstants.Layout.VBOX) //
            .addCssClass(HenshinCss.XMI_NODE) //
            .addArguments(GArguments.cornerRadius(3))
            .add(buildHeader(eObject));
    if(!eObject.eClass().getEAllAttributes().isEmpty()) {
        b.add(createAttributesCompartment(eObject.eClass().getEAllAttributes(), eObject));
    }
    applyShapeData(eObject,b);
    return b.build();
}

private GLabel buildHeader(EObject eObject) {
    return new GLabelBuilder(DefaultTypes.LABEL) //
            .id(UUIDAdapter.toLabelId(eObject))
            .addCssClass(HenshinCss.XMI_LABEL)
            .text(eObject.eClass().getName()) //
            .build();
}

...
\end{lstlisting}


  \subsection{Layouting}
  \label{subsec:layouting}

  \ac{emf} Ecore metamodel files (\textit{.ecore}), Henshin rule files (\textit{.henshin}) and \ac{emf} instance files (\textit{.xmi}), don't contain information about the position or size of elements in a graph. \cite{emf,henshin-repo} To provide a good user experience, the graphical editors need to provide a consistent macro layout for nodes and edges. Newly created nodes should not overlap with existing nodes, and the nodes should stay in the same place after reloading the editor. In general, the \ac{glsp} server is responsible for the macro layouting. \cite{glsp-doc} \ac{glsp} provides multiple options to layout the graph. The interface \code{LayoutEngine} can be used to create a custom layout algorithm, that is applied after the creation of the graphical model from the source model. \ac{glsp} provides the \code{ElkLayoutEngine} implementation, that uses the \ac{elk} to layout the graphical model. \cite{elk-engine} With \ac{elk}, different layout algorithms can be used and additionally configured. Even though \ac{elk} provides much flexibility for the layout, the layout is newly created after every change to the source model. This means that the layout is not consistent and nodes can move around after every change. To provide a consistent layout, the position of nodes need to be stored in addition to the source model. The \ac{glsp} server provides a notation model, that can be used to store the position and size of nodes and edges. \cite{glsp-repo} This brings the overhead of updating the notation model every time when the source model is updated. \ac{glsp} provides classes to make the synchronization of the notation model easier. The notation model is stored in an additional \textit{.notation} file, that is loaded together with the source model and applied to the graphical model in the \code{GModelFactory} using the \code{NotationUtil.applyShapeData(shape, builder)} method. To capture changes of position and size of nodes, the \ac{glsp} client sends the \code{ChangeRoutingPointsOperation} and \code{ChangeBoundsOperation} operations automatically when moving or resizing a node or edge. At the server, the corresponding handlers are updating the notation model using commands to provide undo and redo functionalities.

  To achieve layouting in the \textit{Henshin Web} editor, notation models for the metamodel, Henshin rules, and instances are used. The \textit{.notation} file is created when the source model is loaded for the first time. Here, \ac{elk} can be used to create a fitting initial layout. For the \ac{xmi} instance models, when the graphical model gets created in the \code{GModelFactory}, the shape data from the notation model is added to the \ac{emf} elements over an \ac{emf} \code{Adapter}. Each \ac{emf} \code{EObject} has a list of adapters, that can be used to store additional information. \cite{emf} To connect the notation to an element, the \code{NotationAdapter.getOrAssignNotation()} method checks if the element already has a notation, either returning the existing notation or appending a new Adapter with the notation information.
  For the Henshin rules and the Ecore metamodels, the notation element mapping is stored in the model index that is contained in the \code{ModelState}. The reason for the different indexing approaches and their implementations will be explained in the next section. 


  \subsection{Indexing EMF models}
  \label{subsec:indexing}

  Like the layout information, \ac{emf} Ecore metamodels and \ac{emf} \ac{xmi} instance models don't by default contain unique identifiers for nodes, edges, or attributes. \cite{emf,emf-repo} The graphical model of \ac{glsp} on the other hand uses identifiers for each element that is displayed. If no identifiers are specified when creating the graphical elements, \ac{glsp} generates its own internal unique identifiers. These identifiers are used in edit operations like renaming or deleting a node, where the graphical element needs to be mapped back to the source model element and only the identifier of the graphical element is sent from the client to the server. To be able to map the graphical element back to the source model element, custom identifiers need to be stored. Additionally, during the transformation of the source model into the graphical model, elements need to be accessed multiple times. For example, a source node is accessed over the \ac{emf} package when it is mapped into a \code{GNode} and then again for all its connected edges and attributes. An indexing of the elements avoids multiple lookups in the \ac{emf} source model. To be able to support any created domain meta and instance models and to prevent prerequisites for the use of \ac{emf} models in Henshin Web, the \ac{glsp} server needs to create own indexes for the elements of the source model. 

  The indexing of the three different source model types is implemented in different ways, due to the different internal structures and stored informations. The simplest approach is used for the Henshin rule model. Henshin already creates identifiers for each node and edge of a transformation rule. These identifiers are also stored in the \textit{.henshin} file. When building the graphical model, the identifiers can be accessed over the method \code{getURIFragment(element)} of the \ac{emf} resource. When a new element is created, the index is stored in a bidirectional hash map in the \code{RuleModelIndex} that is accessible over the \code{ModelState}. This index can also be used for the notation model, where the semantic element id needs to be stored to be able to map the layout information back to the source model element. One problem of storing the Henshin identifiers is that a transformation rule is stored as a \ac{lhs} and \ac{rhs} part. Each part has its own identifier, even though it is only one element in the graph. For that Henshin also stores mappings of the \ac{lhs} and \ac{rhs} elements in the \textit{.henshin} file. To be able to correctly map the source model elements to the graphical model elements, these mappings are also stored in the \code{RuleModelIndex}. In listing \ref{lst:rule-indexing} you can see the implementation of the methods \code{getRuleElement(id)} and \code{getRuleElementId(element)} that are used to get the element from the index or get the index of an element. You can see that before searching the index, the mapping list is checked to ensure that the \ac{lhs} element is preferably returned, if it exists. That is for example needed for setting the source and target nodes of an edge. If the edge only appears in the \ac{rhs} part and it should get deleted when applying the rule the \code{getSource()} method returns the \ac{rhs} node element, but the source node was initially created from the \ac{lhs} element. Without the mapping, the source node would not be found in the index and therefore create a new index, that results in an invalid route in the graphical model.

  \begin{lstlisting}[language=TypeScript, caption={Parts of \code{RuleModelIndex}}, label={lst:rule-indexing}]
public void indexRuleElement(String id, GraphElement element) {
    if(ruleElementIndex.containsKey(id))
        return;
    ruleElementIndex.put(id, element);
}

public GraphElement getRuleElement(String id) {
    if (rhsToLhs.containsKey(id)) {
        String lhsId = rhsToLhs.inverseMap().get(id);
        if (ruleElementIndex.containsKey(lhsId)) {
            return ruleElementIndex.get(lhsId);
        }
    }
    return ruleElementIndex.get(id);
}

public String getRuleElementId(GraphElement element) {
    String id = element.eResource().getURIFragment(element);
    if(rhsToLhs.inverseMap().containsKey(id)){
        return rhsToLhs.inverseMap().get(id);
    }
    if(lhrToRhs.inverseMap().containsKey(id)){
        return lhrToRhs.inverseMap().get(id);
    }

    return ruleElementIndex.inverse().get(element);
}
\end{lstlisting}


  This problem doesn't appear for the Ecore metamodel indexing because no content independent indexes are stored in the \ac{emf} model. Here the indexing is used from the existing \ac{glsp} Ecore editor \cite{glsp-ecore-repo}. The \code{EcoreModelIndex} stores an index for the semantic elements, the notation elements and an additional index for inheritance edges.
  For the semantic index, random \acp{uuid} are created. They are used until the client session is closed. During this time, operations on the source model can access \ac{emf} elements by their \acp{uuid} over the stored \code{HashMap} and then apply the operation on the \ac{emf} element. The identifiers are content-independent, which has the advantage, that the identifiers are not changing when nodes are updated. The problem with temporary identifiers on the other hand is, that they cannot be mapped to the source elements after the client session is closed. Therefore, the \acp{uuid} cannot be used in the notation model, because the same notation model needs to be loaded across client sessions. Here, the name of the \ac{emf} class is used, since it is unique for each element in the Ecore metamodel. This index has to be updated if a class is renamed.
  The inheritance index for the Ecore metamodel is used to find already created inheritance edges and retrieve their bend points. With that information, the edges can be connected at bend points to create the typical inheritance arrow structure.

  For the notation models of \ac{xmi} instance models, also content hashes are used as identifiers. Here the name of a object is not unique, because multiple objects of one class can exist. Therefore the content hash, is created from the class name and the names and values of all its attributes. A hash for the class \textit{Client} can look like this: \textit{Client:DynamicEObjectImpl-name:EString=Alice} This content hashes is generated every time the graphical model is created for the first time in a session. It needs to be updated when a attribute value is changed. Content hashes for edges would be even more complex, because they need to include the source and target node hashes combined with the edge type. This is also a reason, why the edge layout information is not stored in the notation model, since the hashes need to be changed for many edit operations to the source model. For \ac{xmi} instance elements, the additional use of adapters is used. The \code{NotationAdapter} and the \code{UUIDAdapter} store the index in the Adapter, which is then directly attached to the \ac{emf} element. This has the advantage, especially for the \code{NotationAdapter}, that when the content hash has to be updated, the notation model can be directly fetched from the \ac{emf} element. It also contains the hashing algorithm for nodes. You can see the implementation of the \code{NotationAdapter} in listing \ref{lst:notation-adapter}. These content hashes need to be used for session independent identifiers, but using them as the only identifier would need a lot of overhead to update the hashes. Therefore, the indexing of the semantic elements works like the metamodel indexing, where \acp{uuid} are used. 
  The combination of the \acp{uuid} and content hashes allows flexibility for editing the source model, while maintaining the connection to the notation model.

  \subsection{Custom \ac{ui} extensions}
  \label{subsec:custom-ui-extensions}

  This section demonstrates the creation of custom \ac{ui} extensions by two different examples. \ac{glsp} provides a predefined interface for creating custom \ac{ui} elements, that could be used in all platform integrations. For that the abstract class \code{Abstract\ac{ui}Extension} must be extended and added to the \textit{henshin-glsp} application module. One simple example is the transformation rule name with its parameters that is displayed in the top left of the rule editor. The extension needs a defined id and a parent container id. With the \code{Set\ac{ui}ExtensionVisibilityAction}, the \ac{ui} element can be made visible from external over the id. With the method \code{initializeContents(containerElement)}, the \ac{html} elements can be created and added to the container. After the model initialization and over a public update method, the class requests the rule name and its parameters over the \code{IActionDispatcher} and updates the \ac{ui}. This update method can be called from any other class, when the \code{RuleName\ac{ui}Extension} is registered and injected over the dependency injection. One example is the explorer view, where the rule can be opened and therefore the rule name must be updated.
  
  This custom explorer is a Theia exclusive extension, accessing the Theia internal \acsp{api}. It cannot be used for other \ac{glsp} platform integrations. To use a custom Theia explorer was already discussed in section \ref{subsec:design-decisions}. To implement a custom explorer, the classes \code{FileNavigatorModel}, \code{FileNavigatorTree}, and \code{FileNavigatorWidget} are extended and registered in the Theia specific \textit{rules-theia} module via dependency injection. To add additional virtual elements in the explorer tree, the two new tree nodes \code{HenshinRootNode}, that contains a list of children, and \code{HenshinRuleNode}, that contains information like the rule name, are created. The method \code{resolveChildren(parent)} is overwritten in the \code{FileNavigatorTree}. Here, if it iterates over a \textit{.henshin} file node, it requests the transformation rules from the server and creates the corresponding \code{HenshinRuleNode} for each rule. It creates also an additional node that works as a \glqq{}add rule\grqq{} button. In the \code{HenshinNavigatorWidget}, the method \code{onSelectionChanged} event is subscribed. It checks if a virtual \code{HenshinRuleNode} was selected. If that is the case, it tries to find the \ac{glsp} rule widget and opens it. It also sends the selected rule name to the server, that is then selecting the rule in the \code{RuleGModelFactory}, where the graphical model is created. To provide a fitting look to the new tree nodes, the \code{HenshinNavigatorWidget} implements the methods \code{toNodeName(node)} and \code{toNodeIcon(node)}. Here, fitting icons are selected and the displayed names are configured.

    % \begin{lstlisting}[language=TypeScript, caption={TypeScript example}, label={lst:ts-example}]
% export class RuleNameUIExtension extends GLSPAbstractUIExtension {
%     static readonly ID = 'rule-name-ui-extension';

%     @inject(EditorContextService)
%     protected editorContext: EditorContextService;

%     @inject(TYPES.IActionDispatcher)
%     protected readonly actionDispatcher: IActionDispatcher;

%     private fullRuleString: string = '';

%     id(): string {
%         return RuleNameUIExtension.ID;
%     }
%     override containerClass(): string {
%         return RuleNameUIExtension.ID;
%     }
%     protected initializeContents(containerElement: HTMLElement): void {
%         containerElement.innerHTML = '';
%         const ruleNameElement = document.createElement('div');
%         ruleNameElement.textContent = this.fullRuleString;
%         ruleNameElement.className = 'rule-name-header';
%         containerElement.appendChild(ruleNameElement);
%     }

%     async postModelInitialization(): Promise<MaybePromise<void>> {
%         await this.setRuleName('');
%     }

%     public async setRuleName(ruleName: string): Promise<void> {
%         if (this.editorContext.diagramType === 'rule-diagram') {
%             var response = await this.actionDispatcher.request<GetParametersOfRuleResponseAction>(
%                 GetParametersOfRuleAction.create(ruleName)
%             );
%             this.fullRuleString =
%                 'Rule ' + response.ruleName + '(' + response.parameters.map(p => p.kind + ' ' + p.name + ':' + p.typeName).join(', ') + ')';

%             if (this.containerElement) {
%                 this.initializeContents(this.containerElement);
%             }
%             this.show(this.editorContext.modelRoot);
%         }
%     }

%     public async updateRuleName(ruleName: string): Promise<void> {
%         this.hide();
%         this.setRuleName(ruleName);
%     }
% }
% \end{lstlisting}



% % \begin{lstlisting}[language=Java, caption={A simple Java method}, label={lst:java-greet}]
% % public class Greeter {
% %     public static String greet(String name) {
% %         return "Hello, " + name + "!";
% %     }

% %     public static void main(String[] args) {
% %         String user = "Flo";
% %         System.out.println(greet(user));
% %     }
% % }
% % \end{lstlisting}
\chapter{Testing and Evaluation}
  \label{sec:testing}

  This chapter discusses the testing strategy of the Henshin Web application. First the general strategy is described. Then the structure and results of the unit tests and end-to-end tests are seperatly presented. Finally, the limitations of the testing are discussed.

  \section{Testing Strategy}
  \label{subsec:testing-strategy}

  For the Java backend, unit tests wer implemented using JUnit. For every milestone of the development process, unit tests were added to ensure the added functionality works as expected. The tests cover the core functionality of the backend. Mocking was used to simulate the behaviour of some components, like the \code{ModelState}.

  To test the UI, automated UI tests were created using Playwright.

  \section{Unit Tests}
  \label{subsec:test-results}

  \section{E2E Tests}
  \label{subsec:performance-evaluation}

  For end-to-end testing, different frameworks were considered to use, including Cypress, Playwrite, Selenium and some more.

  \section{Limitations}
  \label{subsec:user-feedback}
\chapter{Deployment}
\label{sec:deployment}

This chapter shows different deployment and usage options for the Henshin Web editor. The options are evaluated and the best one is selected. Finally, the implementation of the deployment is described.

\section{GLSP Integration Options}
\label{sec:integration-options}
A \ac{glsp} editor can be deployed and used in production in various ways. \ac{glsp} provides platform integrations for the Eclipse Desktop IDE, Eclipse Theia, \ac{vscode}, and as a standalone web application. Each integration brings different integration possibilities, deployment, and usage options for the editor. \cite{glsp-doc} The main considerations for the deployment and usage are:
  \begin{itemize}
    \item The user should need as few dependencies as possible. Dependencies are a browser runtime, an \ac{ide} to install, or an extension to install.
    \item The app should be easy to access. Possible barriers are the creation of an account or the installation of dependencies.
    \item Using a self-hosted server or a cloud service. With a self-hosted server, the user has full access of local files to open and edit. With a cloud service, the user has to upload and download files to the server.
  \end{itemize}
  
  To use \ac{glsp} as a standalone web application, a dependency injection container with the custom \ac{glsp} client is added to a TypeScript browser application. Like that the editor of a certain file as a data source can be displayed. When the app is hosted, no other dependency than a browser runtime is needed to use the standalone diagram editor. \cite{glsp-client-repo} This option provides the most flexibility, as it can be used in any web application, but also requires the most effort to implement, when developing a complete editor. All features, like file management, window management, or other features a \ac{ide} brings, need to be implemented by the developer. \cite{glsp-client-repo} For our use case, the standalone web application is not an option, as these additional features are needed. 

  The other \ac{glsp} integrations are \ac{ide} integrations and therefore provide many features out of the box. For the Eclipse \ac{ide} integration, Eclipse has to be installed, and the \ac{glsp} plugin has to be added to the Eclipse installation. The plugin can be installed from the Eclipse Marketplace or manually by downloading the plugin jar file. \cite{eclipse-doc} The \ac{vscode} integration also provides this option. The \ac{ide} can be installed and the \ac{glsp} editor can be added as an extension. The extension can be installed from the Marketplace or manually using a \textit{.vsix} file. \cite{vscode-doc} The \ac{glsp} \ac{vscode} integration can provide a \textit{.vsix} file. \cite{glsp-repo} \ac{vscode} is the most used \ac{ide}. 73.6\% of developers use \ac{vscode} due to the survey of \citeauthor{stackoverflow2024survey} In \citeyear{stackoverflow2024survey} \cite{stackoverflow2024survey}. An advantage to Eclipse is that \ac{vscode} provides a browser version, which brings the same capabilites as the desktop \ac{ide}. \cite{vscode-doc} So this integration provides the advantage that no \ac{ide} has to be installed to be able to use Henshin Web. The user can open \ac{vscode}, add the extension, and directly open a metamodel, rule, or instance model file and start editing. 

  The Eclipse Theia \ac{ide} is not as widely popular as \ac{vscode} \cite{stackoverflow2024survey}, but its focus is not to provide a ready \ac{ide} but to provide tools to create custom \acsp{ide}. The Eclipse Theia project is part of the Eclipse Foundation and is used as a basis to create your own \acsp{ide} based on web technologies. \cite{theia-doc} They provide the Theia IDE that acts as a template editor and can be downloaded and used on all common operating systems or used in as a web editor in the browser. Due to the focus on providing a framework to build custom \acsp{ide}, Theia provides more options to use extensions and plugins to extend the functionality. You can see the options and their architectural integration into Theia in figure \ref{fig:theia-extensions}.
  \begin{itemize}
    \item \textbf{\ac{vscode} extensions} Theia provides the \ac{vscode} extension \ac{api}, so that existing \ac{vscode} extensions can be used in Theia. They only interact with the \ac{api} and therefore can be installed at runtime.
    \item \textbf{Theia plugins} are working like \ac{vscode} extensions. They interact with the Theia plugin \ac{api} and can also access the \ac{vscode} extension \ac{api}. They can access some Theia specific features, that \ac{vscode} extensions cannot access, like directly contributing to the frontend. They can also be installed at runtime, or be pre-installed at compile time.
    \item \textbf{Headless plugins} are also working like \ac{vscode} extensions. They can also be installed at runtime and can access custom extended Theia backend services.
    \item \textbf{Theia extensions} are the core architecture parts of Theia. Theia is fully built using Theia extensions in a modular way. The template Theia \ac{ide} contains Theia extensions, including the core. Custom Theia extensions can be developed and added to Theia with full access to all Theia functionality via dependency injection. They need to be installed at compile time. \cite{theia-doc}
  \end{itemize}

  The \ac{glsp} Theia integration is creating a Theia extension, that is packed into a custom Theia \ac{ide}. It is also possible to use the \ac{glsp} \ac{vscode} integration that provides a \ac{vscode} extension, that can also be added to a Theia \ac{ide} at runtime. \cite{glsp-repo} The option to use the diagram editor in the browser makes the \ac{glsp} Eclipse integration not interesting for Hensin Web. \ac{vscode} has the advantage of popularity and simplicity to use the editor without any registration or installation. Eclipse Theia has the advantage of modularity and further extensibility. Further features can be added in the future to provide a web-based environment for \ac{mde}. Theia also provides different ways to deploy a Theia \ac{ide}.   These considerations show that the Theia integration is the best option for deploying the Henshin Web editor. Theia combines the advantages of browser-based access, modularity, and extensibility.

  \begin{figure}[h]
    \centering
    \includegraphics[width=0.7\textwidth]{theia-extension}
    \caption{Theia high level extensions and plugins architecture. Image obtained from \cite{theia-doc}}
    \label{fig:theia-extensions}
  \end{figure}


  \section{Deployment Options and Evaluation}
  \label{sec:deployment-evaluation}

  There are different options to provide a \ac{glsp} Theia application. The first option is to host the \ac{glsp} Theia application locally.  For that, \ac{glsp} Theia application can be hosted in a Docker container. \cite{docker} The Docker container can contain the Java server and the TypeScript client, that are started together. The user can then access the editor via a web browser. On a machine with a Docker environment, this solution can be started locally very fast. Compared to options where Henshin Web is hosted in the cloud, this solution provides direct access to the file system. Flaws of this solution are that a Docker environment needs to be installed and set up on the local machine. Also, the user has to start the container manually each time he wants to use the editor and working collaborative is not possible. The Docker container could also be used to deploy the application on a server so that no local docker environment is needed. This second option is supported and implemented with Theia Cloud \cite{theia-cloud-doc}. Theia Cloud is a service by the Eclipse Foundation to deploy Theia based products on Kubernetes clusters \cite{kubernetes} Theia Cloud introduces three custom Kubernetes resource types. \textit{App Definitions} contain all necessary information about the Theia based product. \textit{Workspaces} define persistent storage solutions, where metamodel, rule, or instance model files can be stored for each user. \textit{Sessions} are acting as a runtime representaions. Theia Cloud includes components like a landing page, authentication, authorization, a cloud monitor, and a cloud operator, that deploys sessions and manages workspaces. You can see the different components and their interactions in figure \ref{fig:theia-cloud-components}. The service provides following preconfigured configurations for quickly trying out Theia Cloud on a cluster. \cite{theia-cloud-doc}
  
  For local development and testing purposes, Theia Cloud can be deployed on a local Kubernetes cluster using minikube \cite{minikube-repo}. Minikube is an open-source tool that implements a local Kubernetes cluster on macOS, Linux, and Windows. Its primary goal is to be the best tool for local Kubernetes application development and to support all Kubernetes features that fit. Minikube is an ideal solution for developers who want to host Theia Cloud deployments on a private server or locally before moving to production cloud environments. \cite{minikube-repo} The local deployment eliminates cloud hosting costs during development and testing phases, allows offline development, and provides full control over the Kubernetes cluster configuration. However, it requires the installation of minikube and a container runtime on the developer's machine, and the cluster resources are limited by the local machine's hardware specifications. \cite{minikube-repo} Theia Cloud minikube deployments use VirtualBox as a driver, that also needs to be installed on the local machine. \cite{theia-cloud-minikube}

  For production deployments with high availability, scalability, and enterprise-grade features, Theia Cloud can be deployed on managed Kubernetes services in the cloud. \ac{gke} is Google Cloud's managed Kubernetes service that provides a production-ready platform for running containerized applications at scale. \cite{gke-doc} \ac{gke} manages the entire Kubernetes control plane lifecycle, including automated provisioning, scaling, security patching, and cluster upgrades. This significantly reduces the operational overhead compared to self-managed Kubernetes clusters. \cite{gke-doc} Key features of \ac{gke} include auto-scaling node pools that automatically adjust cluster capacity based on workload demands, built-in security with automated security patching and hardening, integration with Google Cloud's monitoring and logging services. The managed nature of \ac{gke} makes it particularly suitable for production deployments of Theia Cloud, as it provides enterprise-grade reliability, security, and scalability without requiring deep Kubernetes expertise from the development team. However, this comes with cloud hosting costs that scale with resource usage, and requires an active internet connection for cluster management and access.

  \begin{figure}[h]
    \centering
    \includegraphics[width=0.7\textwidth]{theia-cloud-components.png}
    \caption{Interaction between Theia Cloud components. Image obtained from \cite{theia-cloud-doc}}
    \label{fig:theia-cloud-components}
  \end{figure}

  The third option is to use \ac{glsp} Theia application as a desktop application. Theia uses Electron \cite{electron-repo} to bundle the application into a desktop application, that can be installed via an installer. This approach also provides access to the local file system, since the electron application works like a self-hosted web application, and therefore the \ac{glsp} Java server is started locally. All in all, the \ac{glsp} Theia integration provides all different options to use the Henshin Web editor. Further clients can always be added later if needed.

  \begin{table}[h]
  \centering
  \caption{Comparison of GLSP Theia deployment options}
  \label{tab:deployment-comparison}
  \resizebox{\textwidth}{!}{
    \begin{tabular}{|l|p{3.2cm}|p{3.2cm}|p{3.2cm}|p{3.2cm}|}
      \hline
      \textbf{Option} & \textbf{Self-hosted Container} & \textbf{Cloud Hosted Container} & \textbf{Theia Cloud} & \textbf{Desktop (Electron)} \\
      \hline
      \textbf{Installation Effort} & Local Docker Setup required & None (access via browser) & None (creating an account) & Installing over a standard installer \\
      \hline
      \textbf{Dependencies} & Docker runtime, web browser & Web browser only & Web browser only & Application installer \\
      \hline
      \textbf{Multi-user Support} & Single user & Multi-user possible but no shared editor & Built-in shared workspaces, but no shared editor & Single user per installation \\
      \hline
      \textbf{Hosting Requirements} & Local & Cloud service (Container hosting) & Kubernetes cluster & Local machine \\
      \hline
      \textbf{Cross-platform} & Yes (via browser) & Yes (via browser) & Yes (via browser) & Platform-specific builds \\
      \hline
      \textbf{Offline Usage} & Yes & No & No & Yes \\
      \hline
      \textbf{File System Access} & Full local access & Upload/download required & Upload/download required & Full local access \\
      \hline
      \textbf{Costs} & No Costs & Cloud server costs & Cost of Google Cloud Kubernetes cluster & No Costs (maybe provisioning of installer) \\
      \hline
    \end{tabular}
  }
\end{table}

  In table \ref{tab:deployment-comparison} the different deployment options are listed. Each pair of options is similar.The self-hosted Docker and Desktop Electron bring no costs to provide the application, can be used offline and provide full local file access. Here the Desktop option is easier to install and use, as no external dependencies need to be installed before. And even thought it doesn't use the browser, it is based on web technologies. The other two similar options are hosting a container in the cloud and using Theia Cloud. A cloud hosted container would be a self implemented and configured solution. That would bring more flexibility, but also more effort to implement and maintain. Especially when the user management and workspace management should be implemented. Theia Cloud on the other hand provides these features out of the box. Both options bring costs for hosting the server and need an internet connection to be used. They also don't provide access to the local file system, as they are not self-hosted. Since Theia Cloud also leaves configuration options open and it is made for hosting Theia based products, it is the better option to host Henshin Web in the cloud.
  Now the decision is between the Electron desktop application and Theia Cloud. The desktop application has the advantages, that it can be used offline and provides full access to the local file system. This would be no big difference to use Henshin in the Eclipse IDE. Theia Cloud on the other hand provides the advantage of easy access via a web browser without any installation. It also provides the possibility to use Henshin Web on different devices, like tablets or smartphones. Therefore Theia Cloud is the best fitting option to deploy Henshin Web.
  
 \section{Deployment Implementation}
  \label{sec:deployment-implementation}

  This chapter describes the implementation of the deployment of Henshin Web using Theia Cloud. Theia Cloud provides two different deployment options to host the application on. It is possible to host the application on a Kubernetes cluster in the Google Cloud using \ac{gke} or on a local Kubernetes cluster using minikube. The following implementation shows the deployment using \ac{gke}. The implementation using minikube is similar, only the terraform code and the prerequisites of the local machine are different.

  \subsection{Docker Container Architecture}
  \label{subsec:docker-architecture}

  The first thing that needs to be done is to create a docker image of the application. The docker image contains the Java server and the TypeScript client. The resulting docker file is shown in listing \ref{lst:dockerfile}. The Dockerfile implements a multi-stage build approach. This approach separates the build environment from the runtime environment, resulting in smaller and more secure production containers. The build process consists of three distinct stages, each serving a specific purpose in the containerization pipeline. This separation allows for parallel execution of backend and frontend builds, improving overall build performance.

  The first stage, labeled as \textit{build}, establishes the development environment using a Debian Bullseye image with Node.js preinstalled. This stage installs all necessary dependencies for building both the Java backend and TypeScript frontend components. The environment includes Maven for Java compilation, OpenJDK 17 for runtime compatibility, and various system libraries required by Theia. The second stage, \textit{backend}, focuses on compiling the \ac{glsp} Java server. It copies the server source code and uses Maven to clean, compile, and verify the Java components. The Maven settings file is copied to handle GitLab authentication for private repositories, ensuring that all dependencies can be resolved during the build process. The third stage, \textit{frontend}, handles the compilation of all Theia components and plugins. The Yarn autoclean feature is configured to remove unnecessary files like TypeScript sources and test files from the final image, reducing the container size. The final production stage uses a slim Debian image to minimize the attack surface and container size. It installs only the runtime dependencies necessary for Theia operation, including Java runtime, system libraries, and development tools. A non-root user with ID 200 is created, following Theia Cloud standards for user management. The home directory is properly configured with appropriate permissions for file operations. The container exposes port 3000 for web access and configures the entry point to start the Theia backend server with specific parameters for workspace management and plugin loading.

 \subsection{Terraform Configuration}
\label{subsec:terraform-configuration}

When the docker image is built, it can be used to deploy the application on \ac{gke} to make it accessible via the internet for everyone. Theia Cloud provides a modular terraform \cite{terraform-repo} configuration to deploy the Kubernetes cluster to the Google Cloud. This \ac{iac} approach provides the advantage that the infrastructure can be defined as code and therefore can be easily adapted, versioned, and reused.

The Terraform configuration follows a modular architecture with three main components: cluster creation, Helm chart deployment, and Keycloak \cite{keycloak-repo} authentication setup. The main configuration file orchestrates the deployment by calling specialized modules that handle specific infrastructure concerns.

The cluster creation module provisions a production-ready \ac{gke} cluster with auto-scaling node pools. The configuration removes the default node pool to enable custom machine types (\texttt{e2-standard-2}) and scaling parameters (1-2 nodes), providing cost optimization while ensuring adequate resources:

\begin{lstlisting}[language=hcl, caption=GKE Cluster Configuration]
resource "google_container_cluster" "primary" {
  name                     = var.cluster_name
  location                 = var.location
  remove_default_node_pool = true
  initial_node_count       = 1
  deletion_protection      = false
}
\end{lstlisting}

Network connectivity is established through a reserved static IP address that supports both custom domains and automatic DNS resolution via sslip.io. The configuration integrates multiple providers (Google Cloud, Helm, Kubectl, Keycloak \cite{keycloak-repo}) with dynamic authentication using cluster credentials obtained from the GKE module.

The Helm module manages the deployment of essential Kubernetes applications including NGINX Ingress Controller, Cert-Manager for SSL certificates, Keycloak \cite{keycloak-repo} for authentication, PostgreSQL database, and the Theia Cloud application components. Variable management separates configuration from sensitive data, with security-sensitive variables marked appropriately:

\begin{lstlisting}[language=hcl, caption=Variable Configuration]
variable "keycloak_admin_password" {
  description = "Keycloak Admin Password"
  type        = string
  sensitive   = true
}

variable "theia_docker_image" {
  description = "Docker image for your Theia application"
  type        = string
  default     = "gcr.io/henshin-web/henshin-web-model-transformation:latest"
}
\end{lstlisting}

% \subsection{Deployment Execution}
% \label{subsec:deployment-execution}

% The target for the deployment is that it executes automatically in the CI\CD pipeline of the project. Due to restrictions in the used Gitlab instance, the deployment have to be executed manually. Without image-restrictions for the build pipeline, the automatic deployment could be implemented.

% To deploy the project, the prerequisites are that the Google Cloud SDK, Docker, and Terraform are installed on the local machine. The Google Cloud SDK needs to be authenticated with a Google account that has permissions to create and manage \ac{gke} clusters. Docker needs to be installed and running to build the docker image. Terraform needs to be installed to execute the deployment configuration.

% To make the manual deployment easy, a PowerShell script is created that automates the deployment steps. The script first checks if all prerequisites are installed and configured. Then it builds the docker image and pushes it to the Google Container Registry. Finally, it initializes Terraform, plans the deployment, and applies the configuration to deploy the application on \ac{gke}. The script can be found in the project repository \cite{henshin-web-repo}.

\subsection{Deployment Execution}
\label{subsec:deployment-execution}

The deployment of the Henshin Web application follows a systematic process that combines containerization, cloud infrastructure provisioning, and configuration management. The deployment workflow is automated through PowerShell scripts that orchestrate the entire process from local development to production deployment on Google Cloud Platform.

Before executing the deployment, several prerequisites must be satisfied:

\begin{itemize}
    \item Google Cloud Platform account with billing enabled
    \item Google Cloud SDK (gcloud) installed and authenticated
    \item Docker Desktop or Docker Engine installed
    \item Terraform CLI installed (version >= 1.4.0)
    \item Access to the project's Google Container Registry
\end{itemize}

The first step involves building the Docker container and pushing it to Google Container Registry. This process is automated through the \texttt{build-and-push.ps1} script:

\begin{lstlisting}[language=powershell, caption=Container Build Process]
docker build -t "henshin-web-model-transformation:latest" .

$FullImageName = "gcr.io/henshin-web/henshin-web-model-transformation:latest"

docker tag "henshin-web-model-transformation:latest" $FullImageName
docker push $FullImageName
\end{lstlisting}

Once the container image is available in the registry, the infrastructure deployment is executed through the \texttt{deploy.ps1} script. This script starts the creation of the resources in the Google Cloud using Terraform:

\begin{lstlisting}[language=powershell, caption=Infrastructure Deployment]
Set-Location "$PSScriptRoot\..\terraform\configurations\henshin-web-app"

terraform init

$env:GOOGLE_OAUTH_ACCESS_TOKEN = (gcloud auth print-access-token)
terraform apply
\end{lstlisting}

The deployment script automatically handles authentication by obtaining fresh OAuth tokens from the gcloud CLI. This ensures that Terraform has the necessary permissions to create and manage Google Cloud resources without requiring manual token management.

The Terraform configuration executes the deployment in a specific sequence to ensure proper dependency resolution:

\begin{enumerate}
    \item \textbf{GKE Cluster Creation}: Provisions the Kubernetes cluster with auto-scaling node pools
    \item \textbf{Network Configuration}: Reserves static IP addresses and configures ingress rules
    \item \textbf{Core Services Deployment}: Installs NGINX Ingress Controller and Cert-Manager
    \item \textbf{Database Setup}: Deploys PostgreSQL for Keycloak authentication services
    \item \textbf{Authentication Configuration}: Installs and configures Keycloak with realm settings
    \item \textbf{Application Deployment}: Deploys the Henshin Web application and landing page
    \item \textbf{SSL Certificate Provisioning}: Automatically obtains Let's Encrypt certificates
\end{enumerate}

The deployment process uses environment-specific configuration through the \texttt{terraform.tfvars} file, which contains:

\begin{itemize}
    \item Project identification and resource naming conventions
    \item Container image references with specific tags
    \item Authentication credentials for Keycloak and database services
    \item Regional deployment settings and cluster specifications
\end{itemize}

Because this includes sensitive configuration values, they are managed separately from the codebase to maintain security best practices while enabling automated deployment workflows.

Upon successful completion, the deployment script provides the application URL, typically in the format \texttt{https://<static-ip>.sslip.io}, where the application becomes immediately accessible. The automated SSL certificate provisioning through Let's Encrypt ensures secure HTTPS connectivity without manual certificate management.

The deployment can be reversed using the \texttt{destroy.ps1} script, which safely removes all provisioned resources while preserving any persistent data that needs to be retained for future deployments.
\chapter{Usage}
\label{chap:usage}

This chapter provides a user guide with the necessary informations to use Henshin Web. Furthermore, it describes how to set up a development environment for Henshin Web.

\section{User Guide}
\label{sec:user_guide}



\section{Development Setup}
\label{sec:dev_setup}

This section describes how to set up a complete development environment for the Henshin Web project. The project is built using Eclipse \ac{glsp} with a Java server backend and a Theia-based client frontend.

\subsection{Prerequisites}
\label{subsec:prerequisites}

Before setting up the development environment, ensure that the following software components are installed on your system:

\begin{itemize}
    \item \textbf{Node.js} (version $\geq$ 18): Required for building and running the client application
    \item \textbf{Yarn} (version $\geq$ 1.7.0, $<$ 2.x.x): Package manager for JavaScript dependencies
    \item \textbf{Java} (version $\geq$ 17): Required for the GLSP server backend
    \item \textbf{Maven} (version $\geq$ 3.6.0): Build tool for the Java server component
\end{itemize}

Additionally, since the project is built on Eclipse Theia, it is recommended to check the \href{https://github.com/eclipse-theia/theia/blob/master/doc/Developing.md\#prerequisites}{Theia prerequisites} for any platform-specific requirements.

\subsection{Quick Start}
\label{subsec:quick_start}

To get the Henshin Web Model Transformation application up and running quickly, follow these steps:

\subsubsection{Repository Setup}

First, clone the repository and navigate to the project directory:

\begin{lstlisting}[language=bash]
git clone https://gitlab.uni-marburg.de/weidnerf/henshin-web-model-transformation.git
cd henshin-web-model-transformation
\end{lstlisting}

\subsubsection{Building the Application}

Navigate to the source directory and build both client and server components:

\begin{lstlisting}[language=bash]
cd src
yarn build
\end{lstlisting}

This command will:
\begin{itemize}
    \item Build the GLSP server using Maven
    \item Install and build all client dependencies
    \item Prepare the application for execution
\end{itemize}

\subsubsection{Starting the Application}

Once the build is complete, start the application:

\begin{lstlisting}[language=bash]
cd glsp-client
yarn start
\end{lstlisting}

The application will be available at \url{http://localhost:3000}.

\subsection{IDE Setup}
\label{subsec:ide_setup}

The project includes a dedicated Visual Studio Code workspace configuration that provides an optimal development experience. The workspace file \texttt{fmcheck.code-workspace} is located in the \texttt{src} directory.

To set up the VS Code workspace:

\begin{enumerate}
    \item Open Visual Studio Code
    \item Navigate to \texttt{File > Open Workspace from File...}
    \item Select the \texttt{src/fmcheck.code-workspace} file
    \item When prompted, install the recommended extensions for the best development experience
\end{enumerate}

The workspace provides:
\begin{itemize}
    \item Integrated debugging configurations
    \item Recommended extensions for Henshin development
    \item Pre-configured build and run tasks
\end{itemize}

\subsection{Building the Project}
\label{subsec:building}

\subsubsection{Complete Build}

To build all components together:

\begin{lstlisting}[language=bash]
yarn build
\end{lstlisting}

\subsubsection{Individual Component Builds}

Components can be built separately for targeted development:

\begin{lstlisting}[language=bash]
# Build only the client components
yarn build:client

# Build only the server components  
yarn build:server
\end{lstlisting}

\subsubsection{Available VS Code Tasks}

The workspace includes pre-configured tasks accessible via \texttt{Terminal > Run Task...}:

\begin{itemize}
    \item \texttt{Build Henshin GLSP Server}: Builds the Java server component
    \item \texttt{Build Henshin GLSP Client}: Builds the TypeScript client application
    \item \texttt{Watch Henshin GLSP Client}: Enables watch mode for continuous development
\end{itemize}

\subsection{Development and Debugging}
\label{subsec:debugging}

The VS Code workspace provides several launch configurations for debugging different components of the Henshin Web application. These can be accessed through the Run and Debug view (\texttt{Ctrl + Shift + D}):

\begin{itemize}
    \item \textbf{Launch Henshin GLSP Server}: Debug the Java server component with breakpoint support
    \item \textbf{Launch Henshin Theia Backend}: Debug the Theia backend application
    \item \textbf{Launch Theia Frontend}: Debug the browser frontend with Chrome developer tools integration
\end{itemize}

For continuous TypeScript development, enable watch mode to automatically compile files as they are modified:

\begin{lstlisting}[language=bash]
yarn watch
\end{lstlisting}

This functionality is also available as the VS Code task \texttt{Watch Henshin GLSP Client}.

\subsection{Project Structure Overview}
\label{subsec:project_structure}

Understanding the project structure is crucial for effective development. The Henshin Web Model Transformation project is organized into several key components:

\subsubsection{Client Components (\texttt{src/glsp-client/})}

\begin{itemize}
    \item \textbf{henshin-browser-app/}: Main browser application integrating Theia with Henshin-specific plugins
    \item \textbf{henshin-glsp/}: Core diagram client for rendering and user interface modules
    \item \textbf{ecore-theia/}: Ecore model integration for Theia
    \item \textbf{xmi-theia/}: XMI file handling and integration
    \item \textbf{rules-theia/}: Henshin rule-specific Theia extensions
\end{itemize}

\subsubsection{Server Components (\texttt{src/glsp-server/})}

\begin{itemize}
    \item \textbf{src/main/java/}: Core Java server implementation
    \begin{itemize}
        \item \texttt{handler/}: Action handlers for diagram operations
        \item \texttt{model/}: Source model, graphical model, and state management
        \item \texttt{launch/}: GLSP server launcher and configuration
        \item \texttt{palette/}: Custom palette providers for Henshin elements
    \end{itemize}
\end{itemize}

\subsubsection{Henshin SDK Package (\texttt{hensin-package/})}

\begin{itemize}
    \item Pre-packaged Henshin SDK JAR files and dependencies
    \item Maven configuration for Henshin integration
    \item Deployment scripts for package registry
\end{itemize}

\subsection{Container Deployment}
\label{subsec:deployment}

The Henshin Web application supports Docker deployment for consistent environments across different platforms.

\subsubsection{Building the Docker Image}

The project includes automated build scripts:

% \begin{lstlisting}[language=powershell]
% # Using the build script (recommended)
% .\build-and-push.ps1

% # With custom parameters
% .\build-and-push.ps1 -Tag "v1.0.0" -Registry "your-registry.com" -ImageName "henshin-web"
% \end{lstlisting}

Alternatively, build manually:

\begin{lstlisting}[language=bash]
# Build the Docker image
docker build -t henshin-web-model-transformation:latest .

# Tag for registry
docker tag henshin-web-model-transformation:latest gcr.io/henshinwebeditor/henshin-web-model-transformation:latest

# Push to registry
docker push gcr.io/henshinwebeditor/henshin-web-model-transformation:latest
\end{lstlisting}

\subsubsection{Running the Container}

For local development, use the provided script:

% \begin{lstlisting}[language=powershell]
% .\run-container.ps1
% \end{lstlisting}

This will start the container and make the application available at \url{http://localhost:3000}.

For manual container execution:

\begin{lstlisting}[language=bash]
# Run the container
docker run -d -p 3000:3000 --name henshin-web-editor henshin-web-model-transformation:latest

# Check container status
docker ps

# View logs
docker logs henshin-web-editor
\end{lstlisting}


\chapter{Discussion}
  \label{sec:discussion}

  \section{Interpretation of Results}
  \label{subsec:interpretation-results}

  The development of Henshin Web demonstrates a successful implementation of a web-based model transformation editor that addresses the identified accessibility and usability barriers of the traditional Eclipse-based Henshin plugin. The project successfully translated the core functionality of Henshin model transformations into a modern, web-accessible platform built on the \ac{glsp} framework and Eclipse Theia.

  The architectural decisions made throughout the development process have proven effective in achieving the stated research objectives. The choice of Eclipse Theia as the primary platform integration provides the optimal balance between functionality and accessibility, offering a complete \ac{ide} environment without requiring local installation of complex development tools. The modular \ac{glsp} architecture, implementing three distinct diagram modules for \ac{xmi} instances, Henshin rules, and Ecore metamodels, ensures maintainability and extensibility while providing specialized functionality for each model type.

  The implementation successfully addresses the primary research questions posed in this thesis. \textbf{RQ2}, concerning essential functional requirements, is answered through the systematic requirement analysis and implementation of core functionality including graphical editing capabilities, transformation rule application, and seamless integration with existing \ac{emf} ecosystems. The application provides the minimal viable feature set that enables meaningful model transformation work without overwhelming users with unnecessary complexity.

  \textbf{RQ3}, evaluating accessibility and user experience improvements, is demonstrated through the elimination of Eclipse installation requirements and the provision of browser-based access. The web-based approach removes significant barriers to entry, particularly beneficial for educational scenarios and collaborative development environments. The integration of familiar web interface patterns and the reduction of cognitive overhead associated with navigating complex \ac{ide} environments represents a substantial improvement in user accessibility.

  The performance considerations addressed in \textbf{RQ4} reveal both strengths and limitations of the web-based approach. The client-server architecture effectively handles model transformation processing on the server side, leveraging the full capabilities of the Henshin \ac{sdk} while providing responsive graphical interfaces. However, the approach introduces network latency considerations and file management constraints that differ from desktop applications.

  \textbf{RQ5}, concerning integration with existing ecosystems, is successfully addressed through the preservation of standard \ac{emf} file formats and the direct integration of the Henshin \ac{sdk}. The application maintains full compatibility with existing \textit{.ecore}, \textit{.henshin}, and \textit{.xmi} files, ensuring seamless interoperability with established workflows while providing standalone functionality.

  The deployment options evaluation and implementation using Theia Cloud demonstrates the viability of cloud-based model transformation environments. The Infrastructure as Code approach using Terraform provides reproducible deployments while addressing security and scalability concerns through proper authentication and resource management.

  \section{Challenges and Limitations}
  \label{subsec:challenges-limitations}

  Several significant challenges emerged during the development process, providing valuable insights into the complexities of adapting desktop-based modeling tools for web environments. The integration of the Henshin \ac{sdk} into the Maven-based \ac{glsp} project structure required substantial effort to overcome architectural incompatibilities. The Henshin framework's Eclipse plugin architecture necessitated the creation of custom Maven artifacts and dependency management solutions, highlighting the challenges of modernizing legacy academic software for contemporary development workflows.

  The implementation of consistent indexing mechanisms across different \ac{emf} model types presented considerable technical complexity. Each model type (\ac{xmi} instances, Henshin rules, and Ecore metamodels) required distinct indexing strategies due to their different internal structures and identifier management approaches. The \ac{xmi} instance indexing, in particular, required sophisticated content hashing mechanisms to maintain notation model consistency across sessions while accommodating dynamic content changes.

  Layout management and persistence emerged as another significant challenge, as \ac{emf} models inherently lack positional information for graphical elements. The implementation of notation models for each diagram type, combined with the need for consistent macro layout across sessions, required careful synchronization between graphical and semantic models. The decision to use different indexing approaches for different model types, while technically necessary, increases the overall system complexity and maintenance overhead.

  The current implementation exhibits several notable limitations that constrain its applicability in certain scenarios. The web-based architecture introduces file management constraints that differ significantly from traditional desktop applications. Users cannot directly access their local file systems and must rely on upload/download mechanisms when using cloud-hosted deployments. This limitation particularly affects workflows that involve large numbers of files or frequent iteration between different modeling tools.

  The testing strategy, while comprehensive for core functionality, lacks extensive performance evaluation under high-load conditions or with complex model transformations. The focus on unit testing and basic end-to-end testing does not capture potential scalability issues that might emerge in production environments with multiple concurrent users or large model files.

  Documentation and \ac{api} availability for some of the underlying frameworks, particularly Henshin, presented ongoing challenges throughout the development process. The reliance on source code analysis rather than comprehensive documentation slowed development progress and may impact future maintenance efforts. This highlights broader issues in the academic software ecosystem regarding sustainability and developer accessibility.

  The current implementation does not fully replicate all advanced features available in the mature Eclipse Henshin plugin. Complex debugging capabilities, extensive validation mechanisms, and integration with other Eclipse modeling tools remain outside the scope of this implementation. While this limitation is intentional to maintain focus on core accessibility goals, it may limit adoption for users requiring advanced transformation development capabilities.

  \section{Summary of Contributions}
  \label{subsec:summary-contributions}

  This thesis contributes to the field of model-driven engineering through the development of an accessible, web-based alternative to traditional desktop model transformation tools. The primary contributions can be categorized into technical, methodological, and practical domains.

  \textbf{Technical Contributions:} The successful integration of the Henshin model transformation framework with the \ac{glsp} web-based diagramming platform represents a novel technical achievement. The implementation demonstrates how academic modeling tools can be modernized and made accessible through contemporary web technologies while preserving their core functionality. The development of specialized diagram modules for different \ac{emf} model types, each with customized indexing and layout management strategies, provides a blueprint for similar modernization efforts in the model-driven engineering community.

  The creation of custom Maven artifacts from Eclipse plugin components and the subsequent integration into a \ac{glsp} project addresses a common challenge in academic software development: bridging the gap between legacy Eclipse-based architectures and modern development practices. The PowerShell-based packaging solution provides a reusable approach for similar integration challenges.

  The implementation of sophisticated indexing mechanisms for different \ac{emf} model types contributes methodological insights into managing identifier consistency across graphical and semantic model representations. The combination of content hashing for session-independent persistence and UUID-based indexing for session-specific operations provides a robust foundation for web-based model editing applications.

  \textbf{Methodological Contributions:} The systematic requirements analysis approach, distinguishing between core and additional functionality while considering diverse stakeholder needs, provides a methodological framework for similar tool development projects. The evaluation of different platform integration options and deployment strategies offers valuable guidance for academic software projects seeking broader accessibility.

  The Infrastructure as Code approach to deployment, utilizing Terraform and containerization, demonstrates how academic prototypes can be transitioned to production-ready cloud services. The Theia Cloud integration showcases the potential for scalable, multi-user model transformation environments in educational and research contexts.

  \textbf{Practical Contributions:} The resulting Henshin Web application directly addresses identified barriers to model transformation adoption, particularly in educational settings. The elimination of Eclipse installation requirements and the provision of browser-based access significantly reduces the entry barrier for students and researchers exploring model transformation concepts.

  The preservation of full compatibility with existing \ac{emf} file formats ensures that the tool can integrate with established workflows while providing value as a standalone educational and experimental platform. This compatibility maintains the investment in existing model artifacts while extending their accessibility.

  The deployment flexibility, offering both local development and cloud-hosted production options, accommodates different usage scenarios and organizational constraints. The container-based architecture facilitates deployment across various cloud platforms and local environments.

  The comprehensive documentation of development setup, deployment procedures, and user guidance contributes to the sustainability and adoptability of the solution within the academic community. The open-source nature of the implementation ensures that the contributions can be extended and adapted for related projects.

  \section{Suggestions for Future Development}
  \label{subsec:suggestions-future-development}

  The foundation established by Henshin Web creates numerous opportunities for future development that could further enhance its capabilities and extend its impact in the model-driven engineering community. These suggestions are organized by their potential impact and implementation complexity.

  \textbf{Enhanced Collaboration Features:} The web-based architecture provides an ideal foundation for implementing real-time collaborative editing capabilities. Future development could introduce multi-user editing sessions where multiple researchers or students can simultaneously work on transformation rules or model instances. This would involve implementing conflict resolution mechanisms, user awareness indicators, and synchronized change propagation across connected clients. Such features would be particularly valuable in educational settings where instructors could guide students through transformation development in real-time.

  \textbf{Advanced Validation and Analysis:} Building upon the basic model validation currently provided by \ac{glsp}, future versions could integrate more sophisticated analysis capabilities available in the Henshin ecosystem. This includes the implementation of conflict and dependency analysis directly within the web interface, providing visual feedback about potential rule interactions. State space analysis capabilities could be integrated to help users understand the behavior of their transformation rules across different model states.

  \textbf{Integration with External Tools and Platforms:} The web-based architecture facilitates integration with other cloud-based modeling and development tools. Future development could include connectors to version control systems like Git, enabling seamless integration with software development workflows. Integration with cloud storage platforms could address current file management limitations by providing native access to popular file storage services. Additionally, \ac{api} endpoints could be developed to enable integration with other modeling tools and platforms, creating a more connected modeling ecosystem.

  \textbf{Performance Optimization and Scalability Enhancements:} As the platform grows in usage, performance optimization becomes increasingly important. Future development could focus on implementing caching mechanisms for frequently accessed models, optimizing the client-server communication protocols, and developing more efficient serialization strategies for large models. Load balancing and horizontal scaling capabilities could be enhanced to support larger user bases and more computationally intensive transformations.

  \textbf{Extended Platform Support:} While the current implementation focuses on Eclipse Theia integration, future development could expand platform support to include native \ac{vscode} extensions and standalone web applications. Each platform integration offers unique advantages: \ac{vscode} extensions would tap into the large existing user base, while standalone applications could provide more specialized interfaces optimized for specific modeling tasks.

  \textbf{Educational and Training Features:} Given the tool's particular value in educational contexts, future development could include specialized features for learning model transformations. This might include interactive tutorials, example galleries with progressively complex transformation scenarios, and assessment tools that allow instructors to evaluate student understanding of transformation concepts. Integration with learning management systems could facilitate the use of Henshin Web in formal course structures.

  \textbf{Advanced Metamodeling Capabilities:} Future versions could expand beyond basic Ecore editing to include more advanced metamodeling features such as constraint definition using \ac{ocl}, profile mechanisms for domain-specific extensions, and code generation capabilities. Integration with frameworks like Edapt could provide model migration capabilities, allowing users to evolve their metamodels while maintaining instance compatibility.

  \textbf{Community and Ecosystem Development:} The success of Henshin Web could be enhanced through the development of a community ecosystem including transformation rule repositories, best practice guides, and user forums. A marketplace for transformation patterns and templates could accelerate adoption by providing ready-to-use solutions for common modeling scenarios.

  These future development directions would transform Henshin Web from a foundational tool into a comprehensive platform for web-based model-driven engineering. The modular architecture established in this thesis provides a solid foundation for implementing these enhancements while maintaining the core accessibility and usability principles that motivated the original development.


% A. Here argue first why we use continous based (downstream tasks).  Put detailed explanation of methods/ .code/ etc. in Appendix. Show the results, and give recomenddation based on results
% B. Look also at output of other non-cont. methods /baselines. 

% \input{content/Baselines}
% \input{content/Experiments}



% --------------------------------------------
% Backmatter starts here
% --------------------------------------------
\printbibliography{} % bibliography
\appendix
\chapter{Appendix}
  \label{sec:appendix}

  \subsection{Figures}

  \begin{figure}[H]
    \centering
    \includegraphics[width=1\textwidth]{ecore-ui}
    \caption{Henshin Web Ecore graph editor}
    \label{fig:ecore-ui}
  \end{figure}

  \begin{figure}[H]
    \centering
    \includegraphics[width=1\textwidth]{rule-ui}
    \caption{Henshin Web Rules graph editor}
    \label{fig:rule-ui}
  \end{figure}

  \begin{figure}[H]
    \centering
    \includegraphics[width=1\textwidth]{xmi-ui}
    \caption{Henshin Web Instance graph editor}
    \label{fig:xmi-ui}
  \end{figure}

  % User Guide
  \begin{figure}[h]
    \centering
    \includegraphics[width=0.7\textwidth]{attribute-editor}
    \caption{Attribute Editor Window}
    \label{fig:attribute-editor}
  \end{figure}
  \begin{figure}[h]
    \centering
    \includegraphics[width=0.7\textwidth]{operation-editor}
    \caption{Operation Editor Window}
    \label{fig:operation-editor}
  \end{figure}
  \begin{figure}[h]
    \centering
    \includegraphics[width=0.7\textwidth]{reference-editor}
    \caption{Reference Editor Window}
    \label{fig:reference-editor}
  \end{figure}
  \begin{figure}[h]
    \centering
    \includegraphics[width=0.7\textwidth]{enum-editor}
    \caption{Enum Editor Window}
    \label{fig:enum-editor}
  \end{figure}
    \begin{figure}[h]
    \centering
    \includegraphics[width=0.7\textwidth]{rule-editor}
    \caption{Rule Editor Window}
    \label{fig:rule-editor}
  \end{figure}

  \subsection{Code Listings}

    % \begin{lstlisting}[language=Java, caption={Parts of \code{NotationAdapter}}, label={lst:notation-adapter}]
public class NotationAdapter extends AdapterImpl {
    private Shape shape;

    public NotationAdapter(Shape shape) {
        this.shape = shape;
    }

    @Override
    public boolean isAdapterForType(Object type) {
        return type == NotationAdapter.class;
    }

    public static Shape getOrAssignNotation(DynamicEObjectImpl obj) {
        // Return existing Notation if present
        for (var adapter : obj.eAdapters()) {
            if (adapter instanceof NotationAdapter) {
                return ((NotationAdapter) adapter).getShape();
            }
        }

        // Assign new Notation
        String hashId = hashENodeObject(obj);
        Shape shape = notationMap.get(hashId);
        if(shape == null) {
            shape = XMINotationFactory.createNewShape(obj);
        }
        NotationAdapter newAdapter = new NotationAdapter(shape);
        obj.eAdapters().add(newAdapter);

        return shape;
    }

    public static String hashENodeObject(DynamicEObjectImpl eObject) {
        StringBuilder result = new StringBuilder();

        result.append(eObject.eClass().getName()).append(":");
        result.append(DynamicEObjectImpl.class.getSimpleName());

        for (EStructuralFeature feature : eObject.eClass().getEAllStructuralFeatures()) {
            if (feature instanceof EAttribute) {
                result.append("-").append(feature.getName());
                result.append(":").append(feature.getEType().getName());
                result.append("=").append(eObject.eGet(feature).toString());
            }
        }

        return result.toString();
    }

    public static void dispose() {
        notationMap.clear();
    }
}
\end{lstlisting}

    % \begin{lstlisting}[language=Java, caption={Parts of \code{XMIGModelFactory}}, label={lst:gmodel-factory}]
@Override
protected void fillRootElement(GModelRoot newRoot) {
    EGraph instanceNodes = modelState.getInstanceGraph();

    newRoot.getChildren().addAll(instanceNodes.stream()
            .map(eObject -> (DynamicEObjectImpl) eObject) //
            .map(this::createNode)//
            .toList());

    ...
}

public GNode createNode(DynamicEObjectImpl eObject) {
    String type = DefaultTypes.NODE;
    if(eObject.eContainer() == null) {
        type = HenshinTypes.XMI_ROOT_NODE;
    }

    GNodeBuilder b = new GNodeBuilder(type) //
            .id(UUIDAdapter.getOrAssignId(eObject)) //
            .layout(GConstants.Layout.VBOX) //
            .addCssClass(HenshinCss.XMI_NODE) //
            .addArguments(GArguments.cornerRadius(3))
            .add(buildHeader(eObject));
    if(!eObject.eClass().getEAllAttributes().isEmpty()) {
        b.add(createAttributesCompartment(eObject.eClass().getEAllAttributes(), eObject));
    }
    applyShapeData(eObject,b);
    return b.build();
}

private GLabel buildHeader(EObject eObject) {
    return new GLabelBuilder(DefaultTypes.LABEL) //
            .id(UUIDAdapter.toLabelId(eObject))
            .addCssClass(HenshinCss.XMI_LABEL)
            .text(eObject.eClass().getName()) //
            .build();
}

...
\end{lstlisting}




\listoffigures{}
\listoftables{} 

\end{document}