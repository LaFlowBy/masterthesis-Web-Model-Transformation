\documentclass[conference,onecolumn]{IEEEtran}

\usepackage[nolist]{acronym}
\usepackage[backend=bibtex]{biblatex}
\usepackage{graphicx}
\usepackage{hyperref}

\addbibresource{../master-thesis.bib}

\begin{document}

  \title{Proposal: Creating a web-based model transformation UI (with GLSP)}

  \author{\IEEEauthorblockN{Florian Weidner}
    \IEEEauthorblockA{Philipps-University Marburg, Germany\\
      Department of Mathematics and Computer Science, Software engineering group\\
      May 06, 2025\\
  }}

  \maketitle

  \IEEEpeerreviewmaketitle

  \tableofcontents
  \newpage

  \section{Motivation and introduction}
  \label{sec:motivation}

  In software engineering, often \ac{mde} is used to increase development productivity and quality. Concepts are modeled closer to the domain, so that they describe important aspects of a solution with human-friendly abstractions. The models can also be used to generate application fragments, that can be directly used as source code. In the process of \ac{mde}, many activities need to transform source models into different target models, while following a set of transformation rules. This model transformation process is based on algebraic graph transformations. A metamodel is used to model the structure and rules of the concept. The resulting transformation language can provide automatic model creation, development and maintenance activities. \cite{transformations-modeldriven} One framework to use \ac{mde} is \ac{emf} by the Eclipse Foundation. It provides a basis for application development, using modeling and code generation facilities. Much frameworks build upon \ac{emf}, providing various \ac{mde} tools like code generators, graphical diagramming, model transformation or model validation. \cite{emf} One model transformation framework is Henshin. \cite{henshin-repo} It tries to provides model transformation capabilites with a high level of usability. \cite{henshin-usability} For metamodels it uses \ac{emf} Ecore files and for instance models \ac{emf} XMI files. The framework enables transformations on XMI instance files with a defined transformation language. It provides a graphical and textual syntax to create these transformation rules. \cite{henshin-repo} Henshin can be used as a eclipse plugin. Eclipse makes it easy to access, but especially for new users, the heavy editor makes the use of Henshin unintuitive.

  Therefore the goal exists to create a graphical option to use the Henshin model transformations without the overhead of the heavy eclipse editor. A web-based graphical editor would make the use of Henshin even more accessible and intuitive.

  \ac{glsp} is a open-source framework by the Eclipse Foundation to develop custom diagram editors for distributed web-applications. \cite{glsp-repo} It can be used in Eclipse Desktop IDE, Eclipse Theia, Visual Studio Code and embedded in any website. With these fuctionalities, \ac{glsp} fits to create an accessible, intuitive application to create and apply Henshin model transformations.

  \section{Background}
  \label{sec:background}
  Theoretic Background and Frameworks used.

  \subsection{Model transformation}
  \label{subsec:model-checking}
  Model checking is an automatic technique for verifying finite-state reactive systems, such as sequential circuit designs and communication protocols.
  \cite{modelchecking1} \cite{modelchecking2}

  \subsection{Eclipse Foundation}
  \label{subsec:eclipse-foundation}
  The Eclipse Foundation is a not-for-profit, member-supported corporation that hosts the Eclipse open-source community.

  \subsection{\ac{emf}}
  \label{subsec:emf}

  The EMF project is a modeling framework and code generation facility for building tools and other applications based on a structured data model.
  \cite{emf} \cite{emf-repo}

  \subsection{\ac{glsp}}
  \label{subsec:glsp}

  The \ac{glsp} is a framework that allows developers to create graphical user interfaces for web-based diagram editors.
  \cite{glsp-repo}

  \subsection{Henshin}
  \label{subsec:henshin}
  Henshin is an in-place model transformation language for the Eclipse Modeling Framework (EMF).
  \cite{henshin-repo}

  \section{Related Work/Software}
  \label{sec:related-work}

  Similar already existing tools for model checking: Henshin, Groovy...

  \section{Project}
  \label{sec:project-plan}

  \subsection{Requirements}
  \label{subsec:requirements}



  \subsection{Project Plan}
  \label{subsec:project-plan}
    When all requirements are realized, the application should be able to include every use case that you can also do in the Eclipse IDE with the Henshin plugin.

    \begin{itemize}
    \item Create a viewer for \ac{emf} XMI instances
    \item Provide a posibility to display the transformation rules from a henshin file and apply them on the instance model with parameters
    \item Create a viewer for Henshin transformation rules.
    \item Provide editing functionality for transformation rules.
    \item Provide editing functionality for XMI instance models.
    \item Create a viewer for \ac{emf} Ecore metamodels.
    \item Provide editing functionality for Ecore metamodels.
  \end{itemize}

  \subsection{Implementation}
  \label{subsec:implementation}


  \section{Conclusion}


% 1 & Requirements analysis and specification & 1 week & Week 1 \\
%       2 & Design of system architecture & 1 week & Week 2 \\
%       3 & Implementation of EMF XMI instance viewer & 1 week & Week 3 \\
%       4 & Implementation of Henshin transformation rule viewer & 1 week & Week 4 \\
%       5 & Apply transformation rules to instance models & 1 week & Week 5 \\
%       6 & Editing functionality for transformation rules & 1 week & Week 6 \\
%       7 & Editing functionality for XMI instance models & 1 week & Week 7 \\
%       8 & Viewer and editor for Ecore metamodels & 1 week & Week 8 \\
%       9 & Integration and testing & 1 week & Week 9 \\
%       10 & Documentation and final review & 1 week & Week 10 \\

  \printbibliography

\newpage
\section{Acronyms}
\label{sec:acronyms}
\begin{acronym}[AToMPM]
    \acro{glsp}[GLSP]{Graphical Language Server Platform}
    \acro{emf}[EMF]{Eclipse Modeling Framework}
    \acro{mde}[MDE]{Model-Driven Engineering}
    \acro{gui}[GUI]{Graphical User Interface}
    \acro{ide}[IDE]{Integrated Development Environment}
    \acro{sdv}[SDV]{Software-Defined Vehicle}
    \acro{jdt}[JDT]{Java Development Tools}
    \acro{pde}[PDE]{Plug-in Development Environment}
    \acro{sdk}[SDK]{Software Development Kit}
    \acro{api}[API]{Application Programming Interface}
    \acro{uml}[UML]{Unified Modeling Language}
    \acro{xmi}[XMI]{XML Metadata Interchange}
    \acro{xml}[XML]{Extensible Markup Language}
    \acro{lhs}[LHS]{Left-Hand Side}
    \acro{rhs}[RHS]{Right-Hand Side}
    \acro{nac}[NAC]{Negative Application Condition}
    \acro{pac}[PAC]{Positive Application Condition}
    \acro{rpc}[RPC]{Remote Procedure Call}
    \acro{di}[DI]{Dependency Injection}
    \acro{html}[HTML]{Hypertext Markup Language}
    \acro{svg}[SVG]{Scalable Vector Graphics}
    \acro{uri}[URI]{Uniform Resource Identifier}
    \acro{jdk}[JDK]{Java Development Kit}
    \acro{jar}[JAR]{Java Archive}
    \acro{elk}[ELK]{Eclipse Layout Kernel}
    \acro{poc}[POC]{Proof of Concept}
    \acro{uuid}[UUID]{Universally Unique Identifier}
    \acro{atompm}[AToMPM]{A Tool for Multi-Paradigm Modeling}
    \acro{dsml}[DSML]{Domain Specific Modeling Language}
    \acro{vscode}[VS Code]{Visual Studio Code}
  \end{acronym}

\end{document}
