  \chapter{Related Work}
  \label{sec:related-work}

  This chapter reviews some scientific literature and existing web-based modeling tools.

  \section{Scientific Literature}
  \label{subsec:related-scientific-literature}

  This section reviews scientific literature that relates to the development of \textit{Henshin Web}. The discussed works provide implementation guidance and insights for developing similar \ac{glsp} projects.

  \subsection{GLSP-Based Web Modeling Tools}

  In \citeyear{bork2023vision}, \citeauthor{bork2023vision} published a vision for flexible web-based modeling tools built with \ac{glsp} \cite{bork2023vision}. The paper tackles the challenge of creating modern diagram editors that work across different platforms such as web browsers, \ac{vscode}, Eclipse Theia, and the Eclipse desktop \ac{ide}. This is the same challenge Henshin Web addresses. The work explains how \ac{glsp}'s client-server architecture makes modeling tools platform-independent while keeping professional-grade functionality. Web-based modeling tools need to match the quality of traditional desktop applications but can offer additional benefits like easier deployment, platform independence, and collaboration features.
  
  The paper covers architectural patterns for \ac{glsp}-based tools. It shows how to structure server-side model management and client-side rendering. These patterns inform the architecture of Henshin Web. The work also covers extensibility. \ac{glsp} applications can be customized through custom actions, tool palette items, and context menu contributions. Henshin Web uses these features for editing and applying transformation rules. The paper also examines integration with other Eclipse technologies, especially \ac{emf}. For Henshin Web, this integration is necessary because the tool must work with Ecore metamodels and \ac{xmi} instance files. The work confirms that \ac{glsp} is suitable for bringing Henshin to the web while keeping compatibility with the existing Eclipse-based ecosystem.

  \subsection{Practical Experience with GLSP Development}

  \citeauthor{metin2023glsp} documented their experience developing bigUML in \citeyear{metin2023glsp}, a web-based \ac{uml} editor built with \ac{glsp} \cite{metin2023glsp}. The paper reports real-world challenges and solutions when building a production-ready \ac{glsp}-based modeling tool. This makes it relevant for Henshin Web development. The work states architectural decisions. These include organizing the server-side model state, handling concurrent editing scenarios, and implementing complex editing operations across multiple model elements.
  
  The paper examines performance considerations when transforming models into graphical representations for large diagrams. This is relevant when displaying complex Henshin transformation rules or large instance models. The work discusses the learning curve for developers new to \ac{glsp}. It documents common pitfalls and best practices that speed up development. The paper describes strategies to implement custom validation logic, managing undo and redo operations, and integrating with external services. The authors share insights from deploying and operating web-based modeling tools, covering hosting strategies, scalability considerations, and user access management.
  
  The paper also reports observed benefits in practice. New users can access the tool immediately through a web browser without installation, making onboarding easier. Collaboration capabilities improve as well. They help avoid common mistakes and adopt proven patterns for \ac{glsp}-based tool development.

  \subsection{Migration from Eclipse to Web Technologies}

  In \citeyear{domros2018moving}, \citeauthor{domros2018moving} described the migration of the KIELER modeling tool from the Eclipse desktop platform to web technologies using the Theia framework \cite{domros2018moving}. This work is relevant because it addresses the challenge of adapting an existing Eclipse-based modeling tool to run in a web environment. The thesis documents a systematic migration approach. It identifies which components can be reused, which need adaptation, and which require complete reimplementation. KIELER, like Henshin, is an Eclipse plugin that provides graphical modeling capabilities. The migration challenges are therefore comparable.
  
  The thesis predates \ac{glsp}, but KIELER uses Sprotty for diagram rendering. Sprotty is the same underlying \ac{svg}-based diagramming framework that \ac{glsp} builds upon. Later versions of KIELER integrated with \ac{glsp}. This shows the evolution from Sprotty-based custom solutions to \ac{glsp}-based standardized approaches. The work analyzes the architectural differences between Eclipse and Theia. It discusses how the plugin system of Eclipse and extension points map to Theia's modular architecture.
  
  The thesis presents key findings. Separating business logic from \ac{ui} concerns is important for cross-platform compatibility. The work describes strategies for using the Language Server Protocol to share code between different frontend platforms. It evaluates the user experience improvements from the web-based approach. These include faster startup times, reduced installation complexity, and improved accessibility.
  
  The work also identifies challenges. Browser behaviors can conflict with keyboard shortcuts. File system access in a sandboxed web environment needs special handling. Client-server architectures add complexity. These insights provide guidance for the Henshin Web migration strategy. They help decide which Eclipse-specific features to preserve and how to implement them in a web context.

  \subsection{Henshin Framework and Usability}

  \citeauthor{struber2017henshin} documented the state of the Henshin framework in \citeyear{struber2017henshin}, with focus on usability for model transformation development \cite{struber2017henshin}. This work provides essential context for Henshin Web because it documents the features and design philosophy of the framework being adapted for web environments. Henshin provides both graphical and textual syntax for defining transformation rules. This supports different user preferences and use cases. The framework uses algebraic graph transformation foundations. These ensure formal correctness and verifiability of transformations. Henshin Web must preserve these foundations in its web-based implementation.
  
  The paper presents Henshin's current tooling. This includes the Eclipse-based graphical editor for transformation rules, the tree-based editor for transformation units, and the interpreter for executing transformations. Understanding these existing tools is necessary for designing their web-based counterparts. The work emphasizes Henshin's integration with \ac{emf}. Transformation rules are typed over Ecore metamodels. Transformations operate on \ac{xmi} instance files. These are core concepts that Henshin Web must maintain.
  
  The paper discusses usability features. Visual differentiation between preserve, create, and delete actions uses stereotypes. The framework supports \acp{nac} and \acp{pac}. Rules can be parameterized. These features form the baseline functionality that Henshin Web should provide. Users must be able to perform the same transformation tasks as in the Eclipse version. The paper's focus on usability matches Henshin Web's goal of making model transformations more accessible. The web-based interface reduces the barrier to entry.

  \section{Existing Tools and Technologies}
  \label{subsec:related-tools}

    There are many existing tools for model transformations. \citeauthor{kahani2019survey} created a survey in \citeyear{kahani2019survey} of various model transformation tools. They classified 60 different tools, including Henshin. In Figure \ref{fig:tools-environments}, you can see how many tools provide specific execution environments. 73\% of the tools provide plugins for the Eclipse \ac{ide}, and 20\% of the tools are integrated or dependent on other \acsp{ide}. 18\% have no \ac{ide} support, and only two tools are web-based. In total, 89\% of the tools have external dependencies such as an \ac{ide} or other tools. Dependencies often complicate the installation and usage of the tool. \cite{kahani2019survey}

  \begin{figure}[h]
    \centering
    \includegraphics[width=0.6\textwidth]{model-tools.png}
    \caption{Execution environments of model transformation tools. Image obtained from \cite{kahani2019survey}}
    \label{fig:tools-environments}
  \end{figure}

  One web-based tool included in the survey is \ac{atompm} \cite{atompm}. It is a web-based modeling tool to create \ac{dsml} environments, performing model transformations and manipulating and managing models. \cite{atompm} It was created in \citeyear{atompm} and supports all model transformations that are based on T-Core \cite{tcore}, a minimal common basis that allows interoperability between different model transformation languages. \cite{tcore} Metamodels can be defined with a simplified \ac{uml} language. The graphical modeling environment offers debugging and the ability to collaborate and share modeling artifacts in the browser. \cite{atompm}


  There are also other web-based tools for \ac{mde}. WebGME \cite{webGME} is a web-based modeling tool, created in \citeyear{webGME}. It allows to collaboratively design \acp{dsml} using model versioning and broadcasting changes to all active users. It supports prototypical inheritance, where any model can be instantiated recursively, so changes are propagated down the inheritance tree. It also provides scalability, collaborative modeling and model versioning. Metamodels and compositions can be created with WebGME, but no graph transformations can be applied to a model. Even though model transformations are not possible, the editor was one of the first solutions for web-based modeling tools. \cite{webGME} The software provides extension points to customize or extend the software, but no model transformation capabilities were added by any available extension. \cite{webgme-website} The tool is still hosted and maintained, to be used for free. \cite{webgme-website}


  WebDPF \cite{webDPF} is another web-based modeling tool, published in \citeyear{webDPF}. Compared to WebGME and \ac{atompm}, it supports model navigation and element filter capabilities, a JavaScript editor for writing predicate semantics, reusability of transformation rules, partial model completion, and a termination analysis. These features try to improve the usability of the tool. \cite{webDPF} Even though the tool had improvements upon existing tools, the originally mentioned hosted WebDPF portal is offline by now. 


  There is also a \ac{glsp}-based Ecore metamodel editor, created by the \ac{glsp} development team. It was implemented with the \ac{glsp} version 0.9 but never updated further. It allows to create and edit \ac{emf} Ecore models in a Theia web editor. Even though the project cannot be used directly, due to the use of another source model format and breaking changes in major updates of the \ac{glsp} framework, it provides various classes that can be used as a template for the Henshin Web Ecore viewer. One example is the factory code that maps the \ac{emf} Ecore model to the graphical model. \cite{glsp-ecore-repo}
  The findings show, that there are many existing model transformation tools, but only very few web-based solutions, that provide an easy entry into \ac{mde} and model transformations. \textit{Henshin Web} tries to fill this gap.