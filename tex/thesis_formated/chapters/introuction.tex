\chapter{Introduction}
\label{sec:introduction}

\section{Background and Motivation}
\label{subsec:motivation}

In software engineering, often \ac{mde} is used to increase development productivity and quality. \cite{transformations-modeldriven} Concepts are modeled closer to the domain, so that they describe important aspects of a solution with human-friendly abstractions. The models can also be used to generate application fragments, that can be directly used as a template source code. In the process of \ac{mde}, many activities need to transform source models into different target models, while following a set of transformation rules. This model transformation process is based on algebraic graph transformations. A metamodel is used to model the structure and rules of the concept. The resulting transformation language can provide automatic model creation, development, and maintenance activities. \cite{transformations-modeldriven} One framework to use \ac{mde} is \ac{emf} by the Eclipse Foundation. It provides a basis for application development, using modeling and code generation facilities. Many frameworks build upon \ac{emf}, providing various \ac{mde} tools like code generators, graphical diagramming, model transformation, or model validation. \cite{emf} One model transformation framework is Henshin. \cite{henshin-repo} It tries to provide model transformation capabilities with a high level of usability. \cite{henshin-usability} For metamodels it uses \ac{emf} Ecore files. The framework allows to create and apply model transformations on XMI instance files with a defined transformation language. It provides a graphical and textual syntax to create these transformation rules. \cite{henshin-repo} Henshin can be used as a Eclipse plugin. For new users, Eclipse \acs{ide} needs to be installed and the heavy editor makes the use of Henshin unintuitive without prior experience.
Therefore, the goal exists to create a graphical option to use the Henshin model transformations without the overhead of the heavy \acs{ide}, that has to be installed. A web-based graphical editor would make the use of Henshin even more accessible and intuitive.

\ac{glsp} is a open-source framework by the Eclipse Foundation, which can be used to build a web-based Henshin graph editor. The framework is used to develop custom diagram editors for distributed web-applications. \cite{glsp-repo} It can provide graph editors for the Eclipse Desktop IDE, Eclipse Theia, \ac{vscode} and a stanalone version usable in any website. It brings the support of \ac{emf} models as a data source and the functionality of the Henshin SDK can be called from the Java server of \ac{glsp}. \cite{glsp-doc} With these functionalities, \ac{glsp} fits to create an easy accessible, intuitive application to create and apply Henshin model transformations called \textit{Henshin Web}.

\section{Problem Statement}
\label{subsec:problem-statement}

Despite the powerful capabilities of Henshin for model transformations, its current usage presents barriers to adoption and accessibility. The framework is exclusively available as an Eclipse \ac{ide} plugin, which requires users to install and configure the complete Eclipse environment before they can begin working with model transformations. This dependency limits the framework's reach and usability.

First, the requirement for Eclipse installation presents a substantial entry barrier, particularly for newcomers to \ac{mde} who wish to explore model transformation concepts without committing to a full development environment setup. For example, students or researchers who want to quickly experiment with Henshin transformation rules face unnecessary complexity in simply accessing the tool. The installation process, environment configuration, and learning the Eclipse interface adds cognitive overhead that detracts from the Henshin functionality itself.

Second, the Eclipse \ac{ide} presents usability challenges. Eclipse is a heavyweight development environment with many features and a complex interface. It can feel overwhelming when the primary goal is to create and apply model transformations. Users must navigate through multiple perspectives, views, and menus to accomplish basic transformation tasks, leading to reduced productivity and increased frustration.
The standard editor for Ecore, Henshin rules and \ac{XMI} instance files is a tree view, which can get unintuitive for bigger models. In order to edit the transformation rules graphically, a diagram file needs to be initialized separately. To edit the Ecore metamodels, also the diagram has to be initialized specifically. For the \ac{xmi} instance files, an extension has to be installed. Especially the application of transformation rules is not supported graphically even with extensions, but only through a wizzard window.  

Furthermore, the current setup limits collaborative possibilities and portability. Sharing transformation examples across different clients or collaborating on transformation development is not possible with Eclipse. \ac{vcs} have to be used to share files and real-time collaboration or easy access from different devices is not supported.

These accessibility and usability challenges prevent Henshin from reaching its full potential as a model transformation solution, particularly for beginners who face significant initial challenges and among users who could benefit from quick, intuitive access to transformation capabilities without the overhead of a complete \ac{ide} setup.

\section{Research Questions}
\label{subsec:research-questions}

Based on the identified problems with the current Eclipse-based approach to Henshin model transformations, this thesis aims to address the following research questions that guide the development and evaluation of a web-based solution:

\textbf{RQ1: How can Henshin model transformation capabilities be effectively adapted for web-based environments?}
This question investigates the technical feasibility and architectural considerations for translating the desktop-based Henshin functionality into a web application. It examines how the core transformation engine, metamodel handling, and rule definition capabilities can be preserved while adapting to web technologies and browser constraints.


\textbf{RQ2: What are the essential functional requirements for a web-based Henshin editor that maintains usability while reducing complexity?}
This question focuses on identifying the minimum viable feature set that provides meaningful transformation capabilities in order to create an application that can completely handle typical use cases.


\textbf{RQ3: How does a web-based approach improve accessibility and user experience compared to the traditional Eclipse plugin?}
This question evaluates the effectiveness of the web-based solution in addressing the identified barriers to adoption. It examines metrics such as installation complexity, learning curve, collaboration capabilities, and overall user satisfaction when working with model transformations.


\textbf{RQ4: How do different deployment strategies affect the accessibility, usability, and adoption barriers for web-based model transformation tools?}
This question explores various deployment options for the web-based Henshin editor, such as standalone web applications, cloud-hosted services, or integration with existing platforms. It assesses how these strategies impact user access, ease of use, and the overall adoption of the tool among different stakeholder groups.


\textbf{RQ5: How can the web-based editor integrate with existing \ac{emf} and Henshin ecosystems?}
This question explores the compatibility and interoperability requirements for ensuring that the web-based solution can work with existing metamodels, transformation rules, and instance files created in the traditional Eclipse environment, while also providing value as an independent tool.

These research questions collectively address the goal of creating an accessible, intuitive, and functionally adequate web-based alternative to the current Eclipse-dependent Henshin workflow, while ensuring that the solution provides authentic value to the identified stakeholder groups.

\section{Scope and Limitations}
\label{subsec:scope-limitations}

This thesis focuses on developing a web-based solution for Henshin model transformations, with specific boundaries and constraints that define the research scope and acknowledge inherent limitations.

\textbf{Scope of the Research:}
The primary scope encompasses the design, implementation, and evaluation of a web-based editor that provides core Henshin transformation capabilities with the \ac{glsp} framework. The work includes adapting the essential features of the Henshin Eclipse plugin for web environments, focusing on transformation rule creation, metamodel handling, and instance file processing. The implementation targets the fundamental workflow of loading \ac{emf} Ecore metamodels, creating transformation rules through a graphical interface, and applying these transformations to \ac{xmi} instance files.

The research specifically addresses accessibility improvements to minimize the initial challenge for beginners, where users need quick access to model transformation capabilities without extensive setup requirements. The evaluation covers usability aspects, performance characteristics, and functional completeness compared to the traditional Eclipse-based approach. Integration with existing \ac{emf} and Henshin ecosystems is considered to ensure compatibility with established workflows and file formats.

\textbf{Limitations and Constraints:}
Several limitations constrain the scope of this research. The web-based implementation does not aim to replicate every advanced feature available in the mature Eclipse Henshin plugin. Complex transformation scenarios, advanced debugging capabilities are beyond the current scope. The focus remains on core functionality that serves the primary use cases identified in the requirements analysis.

The evaluation methodology is constrained by the availability of test scenarios and user groups within the academic environment. While the research aims to demonstrate improvements over the Eclipse approach, comprehensive studies or extensive industrial validation are outside the scope of this thesis work.

Additionally, the research does not extend to developing new transformation algorithms or enhancing the underlying Henshin transformation engine itself. The focus remains on providing better accessibility and usability for existing Henshin capabilities rather than advancing the theoretical foundations of model transformation techniques.

A system constraint is that the backend has to be Java-based, to be able to directly run the Henshin \acs{sdk}. 

These scope definitions and limitations ensure that the research remains focused and achievable within the constraints of a master's thesis while addressing the core problems identified in current Henshin usage patterns.

\section{Structure of the Thesis}
\label{subsec:structure-thesis}

This thesis is structured as follows, with each chapter building upon the previous ones to provide a comprehensive view of the development and evaluation of the Henshin Web application:

\textbf{Chapter \ref{sec:theoretical-background} - Theoretical Background} introduces the foundational technologies and concepts essential for understanding this work. It covers the Eclipse Foundation ecosystem, \ac{emf} as the modeling framework, Henshin for model transformations, and \ac{glsp} as the web-based graphical editing platform. This chapter establishes the technical context and terminology used throughout the thesis.

\textbf{Chapter \ref{sec:related-work} - Related Work} surveys the landscape of model transformation tools and web-based modeling solutions. It examines scientific literature on model transformation software, analyzes existing tools and their limitations, and compares various web-based modeling environments. This analysis positions Henshin Web within the broader context of available solutions and highlights the gap it aims to fill.

\textbf{Chapter \ref{sec:requirements} - Requirements} defines the functional and non-functional requirements for the Henshin Web editor. It identifies potential user groups, establishes the system scope and context, and details the specific capabilities the application must provide. This chapter serves as the foundation for design and implementation decisions made in subsequent chapters.

\textbf{Chapter \ref{sec:architecture} - Architecture} presents the overall system architecture of Henshin Web. It describes the structural design decisions, component interactions, and architectural patterns employed to meet the identified requirements. The chapter explains how the web-based architecture integrates with existing Henshin and \ac{emf} ecosystems while providing the desired accessibility improvements.

\textbf{Chapter \ref{sec:implementation} - Implementation} details the concrete implementation of core components within the Henshin Web application. It covers the technical realization of key features, integration challenges, and solutions developed to adapt Henshin capabilities for web environments. This chapter provides insight into the practical aspects of translating architectural designs into working software.

\textbf{Chapter \ref{sec:testing} - Testing and Evaluation} discusses the comprehensive testing strategy employed to validate the application's functionality and performance. It covers unit testing approaches, end-to-end testing methodologies, and evaluation criteria used to assess the system's effectiveness. The chapter also addresses testing limitations and their implications for the validation of research outcomes.

\textbf{Chapter \ref{sec:deployment} - Deployment} explores various deployment strategies and options for making Henshin Web accessible to users. It examines different hosting approaches, infrastructure requirements, and considerations for scalability and maintenance. This chapter addresses the practical aspects of delivering the solution to end users.

\textbf{Chapter \ref{chap:usage} - Usage} provides comprehensive user guidance for working with Henshin Web. It includes detailed user guides for creating and editing metamodels, transformation rules, and instance files, as well as administrative guidance for user management and system configuration. This chapter serves as practical documentation for both end users and system administrators.

\textbf{Chapter \ref{sec:discussion-conclusion} - Discussion and Conclusion} synthesizes the research findings, evaluates the success of the approach in addressing the identified problems, and reflects on the broader implications for web-based model transformation tools. It discusses limitations of the current implementation, potential future enhancements, and the contribution of this work to the field of model-driven engineering.