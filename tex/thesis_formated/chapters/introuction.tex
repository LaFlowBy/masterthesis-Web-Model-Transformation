\chapter{Introduction}
\label{sec:introduction}

\section{Background and Motivation}
\label{subsec:motivation}

In software engineering, often \ac{mde} is used to increase development productivity and quality. \cite{transformations-modeldriven} Concepts are modeled closer to the domain, so that they describe important aspects of a solution with human-friendly abstractions. The models can also be used to generate application fragments, that can be directly used as a template source code. In the process of \ac{mde}, many activities need to transform source models into different target models, while following a set of transformation rules. This model transformation process is based on algebraic graph transformations. A metamodel is used to model the structure and rules of the concept. The resulting transformation language can provide automatic model creation, development, and maintenance activities. \cite{transformations-modeldriven} One framework to use \ac{mde} is the \acf{emf} by the Eclipse Foundation. It provides a basis for application development, using modeling and code generation facilities. Many frameworks build upon \ac{emf}, providing various \ac{mde} tools like code generators, graphical diagramming, model transformation, or model validation. \cite{emf} One model transformation framework is Henshin. \cite{henshin-repo} It tries to provide model transformation capabilities with a high level of usability. \cite{henshin-usability} For metamodels it uses \ac{emf} Ecore files. The framework allows to create and apply model transformations on XMI instance files with a defined transformation language. It provides a graphical and textual syntax to create these transformation rules. \cite{henshin-repo} Henshin can be used as a Eclipse plugin. For new users, Eclipse \ac{ide} needs to be installed and the heavy editor makes the use of Henshin unintuitive without prior experience.
Therefore, the goal exists to create a graphical editor to use the Henshin model transformations without the overhead of the heavy \ac{ide}, that has to be installed. A web-based graphical editor would make the use of Henshin even more accessible and intuitive.

\ac{glsp} is an open-source framework by the Eclipse Foundation, which can be used to build a web-based Henshin graph editor. The framework is used to develop custom diagram editors for distributed web-applications. \cite{glsp-repo} It can provide graph editors for the Eclipse Desktop \ac{ide}, Eclipse Theia, \ac{vscode} and a standalone version usable in any website. It brings the support of \ac{emf} models as a data source and the functionality of the existing Henshin \ac{sdk} can be called from the Java server of \ac{glsp}. \cite{glsp-doc} With these functionalities, \ac{glsp} fits to create an easy accessible, intuitive application to create and apply Henshin model transformations called \textit{Henshin Web}. You can find the repository of \textit{Henshin Web} here: \url{https://gitlab.uni-marburg.de/weidnerf/henshin-web-model-transformation}.

\section{Problem Statement}
\label{subsec:problem-statement}

While Henshin offers powerful model transformation capabilities, its integration exclusively as an Eclipse \ac{ide} plugin creates barriers in accessibility for potential users. Users that want to work with Henshin must first install and properly configure the entire Eclipse \ac{ide} environment, a prerequisite that narrows the framework's practical accessibility and constrains the frameworks reach and usability.

The Eclipse installation requirement itself acts as a notable entry barrier. Students and researchers exploring \ac{mde} concepts often want to experiment with transformation rules without investing time in setting up a comprehensive development environment. Yet Henshin's current distribution model forces that. There are complicated installation procedures, environment configurations to overcome and the need to familiarize oneself with Eclipse's interface—all before actually engaging with Henshin's core functionality.

Eclipse as a platform introduces its own complications. As a feature-rich development environment designed for professional software development, Eclipse's extensive capabilities and complex interface can overwhelm users whose objective is creating and executing model transformations. Accomplishing transformation tasks requires navigating through numerous wizards, views, and menu structures, which slows down work and creates friction in the workflow.
The default tree-based editors for Ecore metamodels, Henshin rules, and \ac{xmi} instances become difficult to work with as models grow in size and complexity. For graphical editing of transformation rules, a separate initialization of diagram files has to be manually executed. Similarly, visualizing Ecore metamodels graphically requires specific diagram setup steps. Working with \ac{xmi} instance files demands additional extension installations. Most notably, applying transformation rules lacks any graphical support. Users have to start a wizard to be able to apply model transformations on a \ac{xmi} instance.

Collaboration and flexibility suffer under the current environment. Teams cannot easily share model transformation examples or work together on the same rules within Eclipse. The use of \ac{vcs} brings some colaborative elements, but real-time collaboration or user based workspace access across multiple devices remains impossible.

These accessibility and usability challenges lead to inefficienct workflows. Newcomers encounter steep learning curves before they can productively use the tool, while experienced users who need straightforward access to model transformation features must tolerate unnecessary and unused \ac{ide} complexity.

\section{Research Questions}
\label{subsec:research-questions}

Based on the identified problems with the current Eclipse-based approach to Henshin model transformations, this thesis aims to address the following research questions that guide the development and evaluation of a web-based solution:

\begin{researchquestion}
\label{rq:web-adaptation}
    How can Henshin model transformation capabilities be effectively adapted for web-based environments?
\end{researchquestion}

This question investigates if translating the desktop-based Henshin functionality into a web application is techincally possible and what architecture can fulfill all requirements. It examines how the core model transformation engine, metamodel handling, and rule definition capabilities can be preserved while adapting to web technologies and browser constraints.

\begin{researchquestion}
\label{rq:functional-requirements}
    What are the essential functional requirements for a web-based Henshin editor that maintains usability while reducing complexity?
\end{researchquestion}

This question focuses to identify the minimum viable feature set that is needed to provide meaningful transformation capabilities in order to create an application that can completely handle typical use cases.

\begin{researchquestion}
\label{rq:accessibility-ux}
    How does a web-based approach improve accessibility and user experience compared to the traditional Eclipse plugin?
\end{researchquestion}

This question evaluates the accessibility and usability of the web-based solution. It examines metrics such as installation complexity, learning curve, collaboration capabilities, and overall user satisfaction when working with model transformations.

\begin{researchquestion}
\label{rq:deployment-strategies}
    How do different deployment strategies affect the accessibility, usability, and adoption barriers for web-based model transformation tools?
\end{researchquestion}

This question explores various deployment options for the \textit{Henshin Web} editor. That includes a standalone web applications, cloud-hosted services, or distribution through a desktop application. It assesses how these strategies impact user access, usability, and the overall adoption of the tool among different user groups.

\begin{researchquestion}
\label{rq:ecosystem-integration}
    How can the web-based editor integrate with existing \ac{emf} and Henshin ecosystems?
\end{researchquestion}

This question explores the compatibility and interoperability requirements. It ensures that the web-based solution can work with existing metamodels, transformation rules, and instance files created in the traditional Eclipse environment.

These research questions address the goal of creating an accessible, intuitive, and fully functionally  web-based alternative to the current Eclipse-dependent Henshin workflow.

\section{Scope and Limitations}
\label{subsec:scope-limitations}

This thesis focuses on developing a web-based solution for Henshin model transformations. It contains specific boundaries and constraints that define the research scope. The primary scope includes the design, implementation, and evaluation of a web-based editor using the \ac{glsp} framework. The application should provide the core Henshin model transformation functionality. The work includes adapting the essential features of the Henshin Eclipse \ac{ide} plugin into the web environment. The development focuses on transformation rule creation, metamodel handling, and instance file processing. The implementation should include the workflow of loading \ac{emf} Ecore metamodels, creating transformation rules through a graphical interface, and applying these transformations to \ac{xmi} instance files.

The research addresses accessibility improvements to minimize the initial challenge for beginners, where users need quick access to model transformation capabilities. There they don't need an extensive setup and can directly work with the model graphically. The evaluation covers usability aspects, and functional completeness to cover most meaningful use cases. The system should integrate with the existing \ac{emf} and Henshin ecosystems, to ensure compatibility with established workflows and file formats.

There are some limitations that constrain the scope of this research. On is that the web-based implementation does not aim to replicate every advanced feature from the Eclipse Henshin plugin. Complex transformation scenarios, advanced debugging capabilities are beyond the current scope. The focus remains on core functionality that serves the primary use cases that are defined in the requirements analysis.

The evaluation only covers a limited set of scenarios and user interactions. While the research aims to demonstrate improvements over the Eclipse approach, comprehensive studies or extensive industrial validation are not part of the scope of this thesis work.

Additionally, the research does not extend to develop new transformation algorithms or enhancing the existing Henshin transformation engine. The focus remains on bringing Henshin into the web environment achieving high accessibility and usability, rather than advancing the theoretical foundations of model transformation techniques.

A system constraint is that the backend has to be Java-based, to be able to directly run the Henshin \ac{sdk}. 

These scope definitions and limitations ensure that the research remains focused and achievable within the constraints of a master's thesis while addressing the core problems, that were identified.

\section{Structure of the Thesis}
\label{subsec:structure-thesis}

In this thesis each chapter is building upon the previous ones to provide a full view of the development and evaluation of the \textit{Henshin Web} application. The rest of the thesis is structured as follows:\\

Chapter \ref{sec:background} introduces the basic technologies and concepts that are used to build the application. It covers the Eclipse Foundation ecosystem, \ac{emf} as the modeling framework, Henshin for model transformations, and \ac{glsp} as the web-based graphical editing platform.\\
Chapter \ref{sec:related-work} looks at different model transformation tools and web-based modeling solutions. It presents scientific literature on model transformation software, analyzes existing tools and their limitations, and compares various web-based modeling environments.\\
Chapter \ref{sec:requirements} defines the functional and non-functional requirements for the Henshin Web editor. It identifies potential user groups, defines the system scope and context, and details the specific capabilities the application must provide.\\
Chapter \ref{sec:system-design} presents the overall system architecture of Henshin Web. It describes important design decisions, different component interactions, and describes architectural patterns. \\
Chapter \ref{sec:implementation} shows the concrete implementation of core components within the Henshin Web application. It covers the technical realization of key features, integration challenges, and solutions developed to bring Henshin into the web. \\
Chapter \ref{sec:testing} discusses the testing strategy employed to validate the application's functionality. It describes the unit testing approach, the end-to-end testing environment, and presets test cases to check the application's behavior.\\
Chapter \ref{sec:deployment} compares various deployment strategies and options for making Henshin Web accessible to users. It examines different hosting approaches, infrastructure requirements, and considerations for scalability and maintenance.\\
Chapter \ref{chap:usage} provides a user guide how to use Henshin Web. It includes guides for creating and editing metamodels, transformation rules, and instance files. It also contains an administrative guide for user management and system configuration. This chapter serves as practical documentation for both end users and system administrators.\\
Chapter \ref{sec:conclusion} concludes the research findings, evaluates the success of the approach in addressing the identified problems, and reflects the success of web-based model transformation tools. It discusses limitations of the current implementation, potential future enhancements.