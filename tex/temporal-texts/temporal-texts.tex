\documentclass[conference,onecolumn]{IEEEtran}

\usepackage[nolist]{acronym}
\usepackage[backend=bibtex]{biblatex}
\usepackage{graphicx}
\usepackage{hyperref}

\addbibresource{../master-thesis.bib}

\begin{document}

  \title{Proposal: Web-based Model Checking UI with GLSP}

  \author{\IEEEauthorblockN{Florian Weidner}
    \IEEEauthorblockA{Philipps-University Marburg, Germany\\
      Department of Mathematics and Computer Science, Software engineering group\\
      May 06, 2025\\
  }}

  \maketitle

  \IEEEpeerreviewmaketitle

  \section{Open Questions}

  \begin{itemize}
    \item Welchen Umfang hat die Arbeit? 60-100
    \item Was ist ein guter name
  \end{itemize}

  \section{Meeting Notes}

  Meeting 08.05.2025:
  \begin{itemize}
    \item Wissenschaftliches Arbeiten: Grundlagenkapitel
    \item erster schritt anwendung der regeln auf instance Models
    \item schreiben über grundlagen -> eclipse an sich
    \item henshin interpreter api
    \item 2 stufen: instance transformationen -> regeln spezifizieren
  \end{itemize}

  Meeting 22.05.2025:
  \begin{itemize}
    \item 
  \end{itemize}


  \printbibliography

  \begin{acronym}
    \acro{glsp}[GLSP]{Graphical Language Server Platform}
    \acro{emf}[EMF]{Eclipse Modeling Framework}

  \end{acronym}

\end{document}
