\chapter{Discussion}
  \label{sec:discussion}

  \section{Interpretation of Results}
  \label{subsec:interpretation-results}

  This section evaluates the research findings and addresses the five research questions that guided the development of Henshin Web. The evaluation is based on the successful implementation of a fully functional web-based model transformation editor that demonstrates significant improvements in accessibility and usability.

  \textbf{RQ1: Web-based Adaptation of Henshin Capabilities} - The implementation demonstrates that adapting Henshin for web environments is both technically feasible and practically achievable through the \ac{glsp} framework. The successful integration of the Henshin Java \ac{sdk} into the \ac{glsp} server architecture preserves complete transformation functionality while enabling web delivery. Converting Eclipse plugins into Maven artifacts enabled deployment in standard web application architecture. The three-editor architecture (\code{XMIDiagramModule}, \code{RuleDiagramModule}, \code{EcoreDiagramModule}) successfully handles workflow complexity while maintaining modular design and seamless integration between metamodels, transformation rules, and instance models.

  \textbf{RQ2: Essential Functional Requirements} - The implementation successfully addresses the core use case of applying transformation rules to \ac{emf} \ac{xmi} instance files. The identified requirements prove both necessary and sufficient for meaningful model transformation work across stakeholder groups. The parameter specification system through custom UI extensions maintains transformation flexibility while simplifying application. Integration of rule selection into the Theia explorer enhances workflow efficiency, and undo/redo functionality supports iterative development. The architectural foundation demonstrates scalability for planned enhancements including transformation units and analysis capabilities.

  \textbf{RQ3: Accessibility and User Experience Improvements} - The web-based approach demonstrates significant accessibility improvements. Eliminating Eclipse installation requirements enables browser-based access without complex environment setup, particularly benefiting students and researchers. The graphical interface based on \ac{glsp} and Theia principles provides more intuitive experiences than traditional tree-based editors. Cross-platform accessibility and consistent visual representation reduce learning curves. Direct integration of transformation application into the instance editor improves workflow efficiency compared to Eclipse wizard-based approaches.

  \textbf{RQ4: Deployment Strategy Impact} - The evaluation of multiple deployment options reveals that Theia Cloud proves most effective for maximizing accessibility. The cloud-based approach eliminates installation requirements while providing consistent performance and enterprise-ready capabilities. Trade-offs between Docker containers (local file access vs. installation complexity) and Electron applications (no browser dependency vs. installation requirements) highlight the advantages of the cloud approach. The Kubernetes infrastructure with automatic scaling addresses performance concerns while infrastructure-as-code ensures reproducible deployments.

  \textbf{RQ5: EMF and Henshin Ecosystem Integration} - The implementation demonstrates comprehensive compatibility with existing ecosystems. Direct use of \ac{emf} data structures ensures complete compatibility with existing files, while \code{EMFSourceModelStorage} integration preserves metadata and relationships for seamless round-trip editing. Although notation files use different formats, semantic content remains fully compatible. The Henshin \ac{sdk} integration preserves complete transformation semantics, ensuring identical results between web and Eclipse environments. The modular architecture supports future integration while providing value as an independent tool.

  \section{Challenges and Limitations}
  \label{subsec:challenges-limitations}

  While Henshin Web demonstrates significant progress toward accessible model transformation tools, several challenges emerged during development and certain limitations constrain the system's capabilities compared to the mature Eclipse Henshin plugin.

  \textbf{Technical Integration Challenges} - The integration of Henshin into \ac{glsp} presented substantial challenges. Bridging the architectural gap between Eclipse plugin-based and Maven-based development required creating a custom packaging system to convert 45 Eclipse plugins into Maven artifacts. This introduces maintenance overhead for future Henshin versions. The indexing of \ac{emf} models proved complex, requiring different strategies for each model type: Henshin rules (existing identifiers), Ecore metamodels (UUIDs and content hashes), and \ac{xmi} instances (adapters and content hashes). Notation model synchronization presents ongoing challenges, particularly the content-hash approach for \ac{xmi} instances requiring frequent updates.

  \textbf{Functional Limitations} - The current implementation focuses on core functionality, leaving advanced features unavailable compared to the full Eclipse plugin. Missing capabilities include transformation units for complex scenarios, state space analysis, conflict detection, and comprehensive debugging tools. The web environment constraints prevent integration of step-through debugging capabilities that Eclipse provides. Collaborative editing features, natural for web environments, are not yet implemented, limiting team-based development scenarios.

  \textbf{Performance and Scalability Constraints} - The web architecture introduces performance considerations absent in desktop applications. Client-server communication overhead becomes noticeable with large models, and JSON-RPC protocol serialization creates additional overhead. Current indexing strategies may not scale to very large models, particularly the content-hash approach requiring regeneration for attribute changes. Cloud deployment introduces network dependency, preventing offline work and potentially impacting responsiveness for users with limited connectivity.

  \textbf{Ecosystem and User Experience Limitations} - While maintaining file format compatibility, the system lacks seamless integration with the broader Eclipse modeling ecosystem. Tools depending on Eclipse platform services cannot be directly integrated. Notation model format differences require conversion when moving between environments. The web interface, though more intuitive, cannot provide the full range of keyboard shortcuts and interaction patterns experienced Eclipse users expect. Cloud deployment requires file upload/download, creating workflow friction compared to direct file system access. Current English-only interface limits international adoption.

  \section{Summary of Contributions}
  \label{subsec:summary-contributions}

  This thesis presents several significant contributions to model-driven engineering and web-based modeling tools, successfully demonstrating that sophisticated model transformation capabilities can be made accessible through modern web technologies while maintaining compatibility with established modeling ecosystems.

  \textbf{Web-based Model Transformation Architecture} - The primary contribution is a comprehensive web-based architecture for model transformation editing that bridges traditional desktop-based modeling tools and modern web applications. The three-module architecture (\code{XMIDiagramModule}, \code{RuleDiagramModule}, \code{EcoreDiagramModule}) demonstrates how complex modeling workflows can be decomposed into manageable, specialized components while maintaining integration. The successful integration of the Henshin Java \ac{sdk} into web application architecture shows how existing modeling frameworks can be preserved in new deployment contexts without sacrificing functionality.

  \textbf{Novel GLSP Integration} - The thesis contributes a novel application of the \ac{glsp} framework to model transformation scenarios. While \ac{glsp} has been used for various diagrammatic editing applications, this work demonstrates its applicability to the complex domain of model transformation with multiple interrelated model types. The custom indexing strategies for different \ac{emf} model types represent a significant technical contribution, using hybrid approaches with UUIDs for session-based editing and content hashes for persistent notation. The integration of transformation rule application directly into the graphical interface provides a more intuitive workflow compared to traditional wizard-based approaches.

  \textbf{Accessibility and Barrier Reduction} - A major contribution is the demonstrated reduction in barriers to entry for model transformation technology. By eliminating Eclipse installation and configuration requirements, the web-based approach makes model transformation accessible to broader audiences, particularly students and researchers exploring concepts without extensive tool setup. The user interface design provides a more approachable alternative to traditional tree-based editors, and flexible deployment options demonstrate how model transformation tools can be delivered as accessible services.

  \textbf{Migration Methodology and Open Source Contribution} - The thesis contributes a practical methodology for migrating Eclipse plugin-based tools to web-based architectures. The process of converting Eclipse plugins to Maven artifacts provides a template for similar migration projects. The complete implementation serves as an open-source contribution providing a working example of web-based model transformation editing. The documentation of deployment strategies, including Kubernetes-based infrastructure and Docker containerization, contributes practical knowledge for hosting and scaling web-based modeling tools.

  \textbf{Validation and Foundation for Future Development} - The successful implementation validates the feasibility of web-based approaches for complex modeling tasks traditionally requiring desktop applications. The performance characteristics and scalability analysis contribute empirical data about trade-offs in web-based modeling tool development. The modular architecture and compatibility with existing \ac{emf} and Henshin ecosystems provide a solid foundation for future enhancements while ensuring migration paths for existing users.

  \section{Suggestions for Future Development}
  \label{subsec:suggestions-future-development}

  The foundation established by Henshin Web provides numerous opportunities for enhancement and extension. Future development efforts should focus on expanding transformation capabilities, improving collaborative features, and integrating additional modeling tools to establish Henshin Web as a comprehensive platform for model-driven engineering.

  \textbf{Edapt Integration for Model Evolution} - The integration of Edapt (EMF Adaptation) represents a natural and valuable extension to Henshin Web's capabilities. Edapt provides systematic support for model evolution and migration, addressing the common challenge of maintaining instance models when metamodels change. This integration would involve extending the \code{EcoreDiagramModule} to include Edapt's difference detection algorithms and migration rule generation. The web-based environment would provide unique advantages, allowing users to visualize the impact of metamodel changes across multiple instance files and show proposed migration paths graphically. Users could review and customize migration rules before application, positioning Henshin Web as a complete solution for model evolution scenarios.

  \textbf{Collaborative Modeling Capabilities} - The web-based architecture provides an ideal foundation for implementing collaborative modeling features impossible in traditional desktop applications. Real-time collaborative editing would transform Henshin Web into a platform for team-based model transformation development. Implementation should include operational transformation algorithms for concurrent edits, user presence indicators, visual highlighting of elements being modified by others, and commenting systems for asynchronous collaboration. Version control integration could provide Git-like branching and merging specifically designed for model artifacts, enabling parallel development with systematic integration capabilities. The cloud deployment model facilitates collaboration across organizations and time zones through workspace sharing mechanisms.

  \textbf{Complete Henshin Functionality Implementation} - The current implementation should be systematically extended with the complete Henshin feature set. Transformation units would support sequential, conditional, and iterative rule applications through flowchart-like graphical interfaces for designing complex transformation workflows. State space analysis functionality would enable interactive visualization of transformation paths and resulting models. Conflict and dependency analysis would highlight conflicting rules and visualize dependency relationships. Advanced debugging capabilities should include step-through execution, inspection of intermediate states, and detailed logging with debugging views showing current transformation state, matched elements, and parameter bindings.

  \textbf{Integration with Model-Driven Engineering Ecosystem} - Future development should integrate Henshin Web with the broader MDE ecosystem. Code generation capabilities would extend utility beyond transformation to complete development workflows, allowing users to define generation templates and apply them to transformed models. Integration with model validation frameworks would enable constraint checking, completeness analysis, and semantic validation directly within the web interface. Model repository integration would enable sharing and reuse of metamodels, transformation rules, and instances across projects with version control and dependency management for model artifacts.

  \textbf{Performance and Domain-Specific Extensions} - Performance improvements should address current limitations through optimized indexing with sophisticated caching and incremental updates, lazy loading and virtualization for large models, more efficient serialization protocols, and database integration for scalable storage. Domain-specific extensions could provide specialized modeling notations: business process modeling with BPMN-style notations for transformation workflows, software architecture modeling with UML-style notations and architecture-specific patterns, and educational extensions with guided tutorials, interactive examples, and assessment capabilities including gamification elements and progressive complexity levels for learning support.
