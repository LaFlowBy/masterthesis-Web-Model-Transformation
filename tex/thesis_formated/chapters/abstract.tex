\chapter*{Zusammenfassung}
\addcontentsline{toc}{chapter}{Zusammenfassung}

Henshin ist ein leistungsstarkes Modelltransformations-Framework, das auf dem \ac{emf} aufbaut. Trotz seiner Fähigkeiten steht es vor einer Herausforderung bezüglich der Zugänglichkeit für Benutzer: Es erfordert die vollständige Installation der Eclipse IDE, präsentiert eine komplexe Benutzeroberfläche und bietet keine Unterstützung für kollaborative Entwicklung oder cloud-basierten Zugriff. Diese Hürden schränken die Verbreitung ein, insbesondere unter Studierenden, Forschern und verteilten Teams, die von schnellem Experimentieren ohne erheblichen Einrichtungsaufwand profitieren würden.

Diese Arbeit präsentiert \textit{Henshin Web}, eine umfassende webbasierte Modelltransformations-Anwendung, die traditionelle Einstiegshürden beseitigt und gleichzeitig vollständige Kompatibilität mit dem etablierten \ac{emf} Henshin-Ökosystem beibehält. Aufbauend auf der \acf{glsp} und integriert in Eclipse Theia bietet das System browserbasierte grafische Editoren für \ac{emf} Ecore-Metamodelle, Henshin-Transformationsregeln und XMI-Instanzdateien. Die Architektur verwendet ein Client-Server-Design mit TypeScript-basierten Frontend-Komponenten und einem Java-Backend, das das Henshin SDK direkt integriert und damit semantische Äquivalenz mit Desktop-basierten Henshin Transformationen gewährleistet. Die Implementierung adressiert fünf zentrale Forschungsfragen bezüglich der Machbarkeit der Web-Adaption, essentieller funktionaler Anforderungen, Verbesserungen der Zugänglichkeit, Deployment-Strategien und Ökosystem-Integration. Wesentliche technische Beiträge umfassen eine modulare Drei-Editor-Architektur mit spezialisierten Diagramm-Modulen, angepassten Indexierungsmechanismen für \ac{emf}-Modelle unter Verwendung von \acsp{uuid} und Content-Hashes, benutzerdefinierte \ac{ui}-Erweiterungen für Regelauswahl und Parameterspezifikation sowie ein umfassendes Notationsmodell-Management für persistente Layouts. Die Entwicklung erforderte die Konvertierung von Eclipse-Plugins in Maven-Artefakte, die Implementierung mehrerer Indexierungsstrategien für verschiedene Modelltypen und die Erstellung plattformspezifischer Theia-Erweiterungen bei gleichzeitiger Beibehaltung der Kompatibilität über alle möglichen Client-Integrationen hinweg. Die Evaluierung durch Tests validiert die Funktionalität des Systems und demonstriert Verbesserungen der Zugänglichkeit gegenüber traditionellen Eclipse-basierten Workflows. Unit-Tests decken die Kernfunktionalität des Servers ab, während Playwright-basierte End-to-End-Tests vollständige Benutzer-Workflows einschließlich Modell-Laden, grafischer Bearbeitung und Modelltransformations-Anwendung validieren. Die cloud-basierte Bereitstellung über Theia Cloud eliminiert Installationsanforderungen und bietet sofortigen Browser-Zugriff, wodurch die identifizierten Herausforderungen der Benutzererfahrung adressiert werden. Während sich die aktuelle Implementierung auf Kern-Transformationsfähigkeiten konzentriert, bietet die modulare Architektur eine Grundlage für zukünftige Erweiterungen, einschließlich transformation units, State-Space-Analyse, kollaborativer Bearbeitung durch GLSPs Echtzeit-Synchronisationserweiterungen und Integration zusätzlicher MDE-Tools. Das System demonstriert, dass anspruchsvolle Modelltransformationsfähigkeiten durch moderne Webtechnologien bereitgestellt werden können, ohne funktionale Tiefe zu kompromittieren, und etabliert damit eine neue Option für zugängliche modellgetriebene Engineering-Tools.

\makeatletter

\select@language{english}
\makeatother

\chapter*{Abstract}
\addcontentsline{toc}{chapter}{Abstract}

Henshin is a powerful model transformation framework built on the \acf{emf}. Despite its capabilities, it faces an accessibility challenge for users: it requires full Eclipse IDE installation, presents a complex interface, and lacks support for collaborative development or cloud-based access. These barriers significantly limit adoption, particularly among students, researchers, and distributed teams who would benefit from quick experimentation without substantial setup overhead.

This thesis presents \textit{Henshin Web}, a comprehensive web-based model transformation application that eliminates traditional adoption barriers while maintaining complete compatibility with the established \ac{emf} Henshin ecosystem. Built on the \acf{glsp} and integrated into Eclipse Theia, the system provides browser-accessible graphical editors for \ac{emf} Ecore metamodels, Henshin transformation rules, and XMI instance files. The architecture employs a client-server design with TypeScript-based frontend components and a Java backend that directly integrates the Henshin SDK, ensuring semantic equivalence with desktop-based Henshin transformations.

The implementation addresses five core research questions regarding web adaptation feasibility, essential functional requirements, accessibility improvements, deployment strategies, and ecosystem integration. Key technical contributions include a modular three-editor architecture with specialized diagram modules, custom indexing mechanisms for \ac{emf} models using \acsp{uuid} and content hashes, custom \ac{ui} extensions for rule selection and parameter specification, and comprehensive notation model management for persistent layouts. The development required converting Eclipse plugins into Maven artifacts, implementing multiple indexing strategies for different model types, and creating platform-specific Theia extensions while maintaining compatibility across all possible client integrations.

Evaluation through comprehensive testing validates the system's functionality and demonstrates significant accessibility improvements over traditional Eclipse-based workflows. Unit tests cover the core server functionality, while Playwright-based end-to-end tests validate complete user workflows including model loading, graphical editing, and model transformation application. The cloud-based deployment via Theia Cloud eliminates installation requirements and provides immediate browser access, addressing the identified user experience challenges. While the current implementation focuses on core transformation capabilities, the modular architecture provides a foundation for future enhancements including transformation units, state space analysis, collaborative editing through GLSP's real-time synchronization extensions, and integration with additional MDE tools. The system demonstrates that sophisticated model transformation capabilities can be delivered through modern web technologies without compromising functional depth, establishing a new option for accessible model-driven engineering tools.

\makeatletter

\select@language{english}
\makeatother