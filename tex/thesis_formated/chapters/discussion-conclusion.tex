\chapter{Conclusion}
  \label{sec:conclusion}

  This chapter summarizes the key findings and contributions of this thesis, discusses the challenges encountered during the development of Henshin Web, and presents suggestions for future work to enhance and extend the application.

  \section{Interpretation of Results}
  \label{subsec:interpretation-results}

  This section evaluates the resulting software and the research and addresses the five research questions that guided the development of Henshin Web.

  The implementation demonstrates that adapting Henshin into a web environment is both technically feasible and practically achievable through the \ac{glsp} framework. The successful integration of the Henshin \ac{sdk} into the \ac{glsp} server architecture preserves complete transformation functionality while providing a web-based application through Theia Cloud. Converting Eclipse plugins into Maven artifacts enabled the inclusion of the Henshin code into the \ac{glsp} project architecture. The three-editor architecture (\code{XMIDiagramModule}, \code{RuleDiagramModule}, \code{EcoreDiagramModule}) successfully handles complex workflows while maintaining modular design and issue-free integration between metamodels, transformation rules, and instance models. This successfully answers research question \ref{rq:web-adaptation}.

  Research question \ref{rq:functional-requirements} was also answered after implementing and testing \textit{Henshin Web}. The application successfully addresses the core use case of creating a metamodel and creating and applying transformation rules to \ac{emf} \ac{xmi} instance files of the metamodel. The identified requirements prove both necessary and fitting for a meaningful use of the software. The integration of rule selection into the Theia explorer and the undo and redo functionality enhances workflow efficiency. The architecture also allows future extensions of the software and provides a scalable environment for additional use cases.

  Research question \ref{rq:accessibility-ux} is about accessibility improvements from a web-based approach. Eliminating Eclipse installation requirements enables browser-based access without complex environment setup, particularly benefiting new users of Henshin. The graphical interface based on \ac{glsp} and Theia principles provides an intuitive and widely used environment. Cross-platform accessibility, consistent visual representation and the usage of only relevant \ac{ui} elements reduce learning curves. The direct integration of model transformation application into the instance editor improves workflow efficiency compared to Eclipse wizard-based approach. With the cloud deployment, users are dependent on a stable internet connection to get an optimal experience.

  For research question \ref{rq:deployment-strategies}, the evaluation of multiple deployment options reveals that Theia Cloud is the best choice to maximize accessibility. The cloud-based approach eliminates installation requirements while providing consistent performance and a fully functional scope for all requirements. Trade-offs between self-hosted Docker containers or Electron applications with local file access but installation complexity highlight the advantages of the cloud approach. The deployment on a Kubernetes infrastructure with automatic scaling provides scalable performance while infrastructure-as-code ensures reproducible deployments.

  The implementation demonstrates comprehensive compatibility with the existing Henshin ecosystem. That answers research question \ref{rq:ecosystem-integration}. The direct use of \ac{emf} data structures ensures complete compatibility with existing files, while the \code{EMFSourceModelStorage} integration preserves metadata and relationships for \ac{glsp} internal features. Although notation files use different formats, the semantic content remains fully compatible. Existing projects can be migrated by uploading or copying the files into the Theia explorer, having at most one metamodel and henshin rule files. The Henshin \ac{sdk} integration preserves complete transformation semantics, ensuring identical results between web and Eclipse environments. The modular architecture supports future integrations all fitting in the \ac{emf} ecosystem.

  \section{Challenges and Limitations}
  \label{subsec:challenges-limitations}

  The interpretation showed good results regarding the research questions. Still, several challenges came up during the development and certain limitations remain in the current implementation.

  There were several technical integration challenges. The main one was the integration of the Henshin \ac{sdk} into \ac{glsp}, which presented some challenges. Bridging the architectural gap between Eclipse plugin-based and Maven-based development required creating a custom packaging system to convert the Henshin Eclipse plugins into Maven artifacts. This introduces maintenance overhead for future Henshin versions, where the new Eclipse based Henshin executables have to be mapped manually. The indexing of \ac{emf} models couldn't be achieved by a single approach. Different model structures required different strategies for each model type: Henshin rules had existing identifiers, Ecore metamodels uses \acp{uuid} and content hashes, and \ac{xmi} instances use adapters and content hashes. The synchronization of the notation models brings ongoing challenges, particularly the content-hash approach for \ac{xmi} instances requires frequent updates of the notation files. 

  The current implementation has some functional limitations focusing only on the core functionality of Henshin. It leaves advanced features unavailable for now. Missing capabilities are transformation units for complex scenarios, the usage of JavaScript variables, rule matching, state space analysis, conflict detection, and comprehensive debugging tools. Also, collaborative editing features are not yet implemented, limiting team-based development scenarios.

  The system also brings some smaller ecosystem and user experience limitations. Although users are not required to install an \ac{ide}, a user registration is necessary to use Henshin Web. While maintaining file format compatibility, the Theia application lacks integration with the broader Eclipse modeling ecosystem. Tools depending on Eclipse platform services cannot be directly integrated. Henshin Web allows the extension of different \ac{mde} use cases in the future, but they have to be develop explicitly. Due to notation model format differences, the layout of graphical elements is not consistent when moving between environments. Cloud deployment requires file upload/download, creating workflow friction compared to direct file system access. The current interface is only available in English, which limits international adoption.

  \section{Summary of Contributions}
  \label{subsec:summary-contributions}

  The primary contribution of this thesis is the successful creation of a working, fully web-based model transformation application that reduces barriers to entry the usage of model transformations with Henshin while maintaining functional compatibility with the established Eclipse-based ecosystem. Henshin Web shows that model transformation capabilities can be made accessible through modern web technologies. The application enables users to create metamodels, define model transformation rules, and apply model transformations to instance models entirely through a web browser. The web-based implementation preserves the complete Henshin Java \ac{sdk} functionality and maintains full compatibility with existing \ac{emf} Ecore metamodels, \ac{xmi} instance files, and Henshin transformation rules. This ensures that users can transition between Eclipse-based and web-based workflows. Additionally, a testing environment with backend unit tests and frontend end-to-end tests was established to ensure software quality. The work also provides an analysis of different deployment options for \ac{glsp} projects. 
  The user guide was created to help new users get started with Henshin Web. Additionally, the development and deployment guide helps future developers set up the project and deploy their improvements or extensions on their own infrastructure.

  \section{Suggestions for Future Development}
  \label{subsec:suggestions-future-development}

  The scalable and extendable architecture of Henshin Web provides numerous opportunities for enhancement and extension. The modularity of Theia is a good basis for extending the application further in the future. Additional Theia modules can be integrated to cover additional \ac{mde} use cases. Code generation capabilities would provide utility to allow complete development workflows, allowing users to define the transformation rules as a model and then generating a basis for the development of the application. Integration with model validation frameworks would enable constraint checking, completeness analysis, and semantic validation directly within the web interface. Therefore, Henshin Web can be used as an architectural template for various \ac{mde} use cases in the future.
  But Henshin Web should also be extended to provide the full Henshin functionality. The current implementation should be systematically extended with the complete Henshin feature set. Transformation units would support sequential, conditional, and iterative rule applications to be able to design complex transformation workflows. State space analysis functionality would enable interactive visualization of transformation paths and resulting models. Conflict and dependency analysis would highlight conflicting rules and visualize dependency relationships. Advanced debugging capabilities should include step-through execution, inspection of intermediate states, and detailed logging with debugging views showing current transformation state, matched elements, and parameter bindings.

  An ongoing problem is the adoption of instance files after updating the Ecore metamodel. The integration of Edapt (\ac{emf} Adaptation) \cite{edapt-repo} provides systematic support for model evolution and migration, addressing the common challenge of maintaining instance models when metamodels change. This integration would involve extending the \code{EcoreDiagramModule} to include Edapt's difference detection algorithms and migration rule generation. Since Henshin Web projects often contain multiple instance files that should always be in sync with the metamodel, inconsistencies may arise when the instance files are not updated after metamodel changes. Edapt can help to overcome these inconsistencies by automatically migrating the instance files without requiring manual intervention from the user and without custom migration scripts.

  Another potential enhancement is real-time collaborative editing. This feature would transform Henshin Web into a platform for team-based model transformation development. \ac{glsp} provides a real-time synchronous diagram collaboration extension \cite{glsp-collab} that is compatible with the existing architecture and Eclipse Theia. It can be integrated into Henshin Web to enable multiple users to collaboratively edit metamodels, transformation rules, and instance models simultaneously, enhancing teamwork and knowledge sharing. It can be used without major architectural changes because one user acts as a coordinator communicating with the \ac{glsp} server, as shown in Figure \ref{fig:glsp-collaboration}.

  \begin{figure}
    \centering
    \includegraphics[width=0.7\textwidth]{glsp-collab}
    \caption{GLSP Real-Time Collaboration Extension \cite{glsp-collab}}
    \label{fig:glsp-collaboration}
  \end{figure}
