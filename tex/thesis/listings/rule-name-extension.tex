% \begin{lstlisting}[language=TypeScript, caption={TypeScript example}, label={lst:ts-example}]
% export class RuleNameUIExtension extends GLSPAbstractUIExtension {
%     static readonly ID = 'rule-name-ui-extension';

%     @inject(EditorContextService)
%     protected editorContext: EditorContextService;

%     @inject(TYPES.IActionDispatcher)
%     protected readonly actionDispatcher: IActionDispatcher;

%     private fullRuleString: string = '';

%     id(): string {
%         return RuleNameUIExtension.ID;
%     }
%     override containerClass(): string {
%         return RuleNameUIExtension.ID;
%     }
%     protected initializeContents(containerElement: HTMLElement): void {
%         containerElement.innerHTML = '';
%         const ruleNameElement = document.createElement('div');
%         ruleNameElement.textContent = this.fullRuleString;
%         ruleNameElement.className = 'rule-name-header';
%         containerElement.appendChild(ruleNameElement);
%     }

%     async postModelInitialization(): Promise<MaybePromise<void>> {
%         await this.setRuleName('');
%     }

%     public async setRuleName(ruleName: string): Promise<void> {
%         if (this.editorContext.diagramType === 'rule-diagram') {
%             var response = await this.actionDispatcher.request<GetParametersOfRuleResponseAction>(
%                 GetParametersOfRuleAction.create(ruleName)
%             );
%             this.fullRuleString =
%                 'Rule ' + response.ruleName + '(' + response.parameters.map(p => p.kind + ' ' + p.name + ':' + p.typeName).join(', ') + ')';

%             if (this.containerElement) {
%                 this.initializeContents(this.containerElement);
%             }
%             this.show(this.editorContext.modelRoot);
%         }
%     }

%     public async updateRuleName(ruleName: string): Promise<void> {
%         this.hide();
%         this.setRuleName(ruleName);
%     }
% }
% \end{lstlisting}



% % \begin{lstlisting}[language=Java, caption={A simple Java method}, label={lst:java-greet}]
% % public class Greeter {
% %     public static String greet(String name) {
% %         return "Hello, " + name + "!";
% %     }

% %     public static void main(String[] args) {
% %         String user = "Flo";
% %         System.out.println(greet(user));
% %     }
% % }
% % \end{lstlisting}