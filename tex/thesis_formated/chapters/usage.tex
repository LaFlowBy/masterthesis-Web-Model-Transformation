\chapter{Usage}
\label{chap:usage}

This chapter provides a user guide with the necessary informations to use Henshin Web. Furthermore, it describes how to set up a development environment for Henshin Web.

\section{User Guide}
\label{sec:user_guide}



\section{Development Setup}
\label{sec:dev_setup}

This section describes how to set up a complete development environment for the Henshin Web project. The project is built using Eclipse \ac{glsp} with a Java server backend and a Theia-based client frontend.

\subsection{Prerequisites}
\label{subsec:prerequisites}

Before setting up the development environment, ensure that the following software components are installed on your system:

\begin{itemize}
    \item \textbf{Node.js} (version $\geq$ 18): Required for building and running the client application
    \item \textbf{Yarn} (version $\geq$ 1.7.0, $<$ 2.x.x): Package manager for JavaScript dependencies
    \item \textbf{Java} (version $\geq$ 17): Required for the GLSP server backend
    \item \textbf{Maven} (version $\geq$ 3.6.0): Build tool for the Java server component
\end{itemize}

Additionally, since the project is built on Eclipse Theia, it is recommended to check the \href{https://github.com/eclipse-theia/theia/blob/master/doc/Developing.md\#prerequisites}{Theia prerequisites} for any platform-specific requirements.

\subsection{Quick Start}
\label{subsec:quick_start}

To get the Henshin Web Model Transformation application up and running quickly, follow these steps:

\subsubsection{Repository Setup}

First, clone the repository and navigate to the project directory:

\begin{lstlisting}[language=bash]
git clone https://gitlab.uni-marburg.de/weidnerf/henshin-web-model-transformation.git
cd henshin-web-model-transformation
\end{lstlisting}

\subsubsection{Building the Application}

Navigate to the source directory and build both client and server components:

\begin{lstlisting}[language=bash]
cd src
yarn build
\end{lstlisting}

This command will:
\begin{itemize}
    \item Build the GLSP server using Maven
    \item Install and build all client dependencies
    \item Prepare the application for execution
\end{itemize}

\subsubsection{Starting the Application}

Once the build is complete, start the application:

\begin{lstlisting}[language=bash]
cd glsp-client
yarn start
\end{lstlisting}

The application will be available at \url{http://localhost:3000}.

\subsection{IDE Setup}
\label{subsec:ide_setup}

The project includes a dedicated Visual Studio Code workspace configuration that provides an optimal development experience. The workspace file \texttt{fmcheck.code-workspace} is located in the \texttt{src} directory.

To set up the VS Code workspace:

\begin{enumerate}
    \item Open Visual Studio Code
    \item Navigate to \texttt{File > Open Workspace from File...}
    \item Select the \texttt{src/fmcheck.code-workspace} file
    \item When prompted, install the recommended extensions for the best development experience
\end{enumerate}

The workspace provides:
\begin{itemize}
    \item Integrated debugging configurations
    \item Recommended extensions for Henshin development
    \item Pre-configured build and run tasks
\end{itemize}

\subsection{Building the Project}
\label{subsec:building}

\subsubsection{Complete Build}

To build all components together:

\begin{lstlisting}[language=bash]
yarn build
\end{lstlisting}

\subsubsection{Individual Component Builds}

Components can be built separately for targeted development:

\begin{lstlisting}[language=bash]
# Build only the client components
yarn build:client

# Build only the server components  
yarn build:server
\end{lstlisting}

\subsubsection{Available VS Code Tasks}

The workspace includes pre-configured tasks accessible via \texttt{Terminal > Run Task...}:

\begin{itemize}
    \item \texttt{Build Henshin GLSP Server}: Builds the Java server component
    \item \texttt{Build Henshin GLSP Client}: Builds the TypeScript client application
    \item \texttt{Watch Henshin GLSP Client}: Enables watch mode for continuous development
\end{itemize}

\subsection{Development and Debugging}
\label{subsec:debugging}

The VS Code workspace provides several launch configurations for debugging different components of the Henshin Web application. These can be accessed through the Run and Debug view (\texttt{Ctrl + Shift + D}):

\begin{itemize}
    \item \textbf{Launch Henshin GLSP Server}: Debug the Java server component with breakpoint support
    \item \textbf{Launch Henshin Theia Backend}: Debug the Theia backend application
    \item \textbf{Launch Theia Frontend}: Debug the browser frontend with Chrome developer tools integration
\end{itemize}

For continuous TypeScript development, enable watch mode to automatically compile files as they are modified:

\begin{lstlisting}[language=bash]
yarn watch
\end{lstlisting}

This functionality is also available as the VS Code task \texttt{Watch Henshin GLSP Client}.

\subsection{Project Structure Overview}
\label{subsec:project_structure}

Understanding the project structure is crucial for effective development. The Henshin Web Model Transformation project is organized into several key components:

\subsubsection{Client Components (\texttt{src/glsp-client/})}

\begin{itemize}
    \item \textbf{henshin-browser-app/}: Main browser application integrating Theia with Henshin-specific plugins
    \item \textbf{henshin-glsp/}: Core diagram client for rendering and user interface modules
    \item \textbf{ecore-theia/}: Ecore model integration for Theia
    \item \textbf{xmi-theia/}: XMI file handling and integration
    \item \textbf{rules-theia/}: Henshin rule-specific Theia extensions
\end{itemize}

\subsubsection{Server Components (\texttt{src/glsp-server/})}

\begin{itemize}
    \item \textbf{src/main/java/}: Core Java server implementation
    \begin{itemize}
        \item \texttt{handler/}: Action handlers for diagram operations
        \item \texttt{model/}: Source model, graphical model, and state management
        \item \texttt{launch/}: GLSP server launcher and configuration
        \item \texttt{palette/}: Custom palette providers for Henshin elements
    \end{itemize}
\end{itemize}

\subsubsection{Henshin SDK Package (\texttt{hensin-package/})}

\begin{itemize}
    \item Pre-packaged Henshin SDK JAR files and dependencies
    \item Maven configuration for Henshin integration
    \item Deployment scripts for package registry
\end{itemize}

\subsection{Container Deployment}
\label{subsec:deployment}

The Henshin Web application supports Docker deployment for consistent environments across different platforms.

\subsubsection{Building the Docker Image}

The project includes automated build scripts:

% \begin{lstlisting}[language=powershell]
% # Using the build script (recommended)
% .\build-and-push.ps1

% # With custom parameters
% .\build-and-push.ps1 -Tag "v1.0.0" -Registry "your-registry.com" -ImageName "henshin-web"
% \end{lstlisting}

Alternatively, build manually:

\begin{lstlisting}[language=bash]
# Build the Docker image
docker build -t henshin-web-model-transformation:latest .

# Tag for registry
docker tag henshin-web-model-transformation:latest gcr.io/henshinwebeditor/henshin-web-model-transformation:latest

# Push to registry
docker push gcr.io/henshinwebeditor/henshin-web-model-transformation:latest
\end{lstlisting}

\subsubsection{Running the Container}

For local development, use the provided script:

% \begin{lstlisting}[language=powershell]
% .\run-container.ps1
% \end{lstlisting}

This will start the container and make the application available at \url{http://localhost:3000}.

For manual container execution:

\begin{lstlisting}[language=bash]
# Run the container
docker run -d -p 3000:3000 --name henshin-web-editor henshin-web-model-transformation:latest

# Check container status
docker ps

# View logs
docker logs henshin-web-editor
\end{lstlisting}

