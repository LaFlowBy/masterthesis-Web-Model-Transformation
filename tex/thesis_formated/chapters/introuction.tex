\chapter{Introduction}
\label{sec:introduction}

\section{Background and Motivation}
\label{subsec:motivation}

In software engineering, often \ac{mde} is used to increase development productivity and quality. \cite{transformations-modeldriven} Concepts are modeled closer to the domain, so that they describe important aspects of a solution with human-friendly abstractions. The models can also be used to generate application fragments, that can be directly used as a template source code. In the process of \ac{mde}, many activities need to transform source models into different target models, while following a set of transformation rules. This model transformation process is based on algebraic graph transformations. A metamodel is used to model the structure and rules of the concept. The resulting transformation language can provide automatic model creation, development, and maintenance activities. \cite{transformations-modeldriven} One framework to use \ac{mde} is \ac{emf} by the Eclipse Foundation. It provides a basis for application development, using modeling and code generation facilities. Many frameworks build upon \ac{emf}, providing various \ac{mde} tools like code generators, graphical diagramming, model transformation, or model validation. \cite{emf} One model transformation framework is Henshin. \cite{henshin-repo} It tries to provide model transformation capabilities with a high level of usability. \cite{henshin-usability} For metamodels it uses \ac{emf} Ecore files and for instance models \ac{emf} XMI files. The framework enables transformations on XMI instance files with a defined transformation language. It provides a graphical and textual syntax to create these transformation rules. \cite{henshin-repo} Henshin can be used as a Eclipse plugin. Eclipse makes it easy to access, but especially for new users, the heavy editor makes the use of Henshin unintuitive.
Therefore, the goal exists to create a graphical option to use the Henshin model transformations without the overhead of the heavy Eclipse editor. A web-based graphical editor would make the use of Henshin even more accessible and intuitive.

\ac{glsp} is a open-source framework by the Eclipse Foundation, which can be used to build a web-based Henshin graph editor. The framework is used to develop custom diagram editors for distributed web-applications. \cite{glsp-repo} It can provide graph editors for the Eclipse Desktop IDE, Eclipse Theia, \ac{vscode} and a stanalone version usable in any website. It brings the support of \ac{emf} models as a data source and the Henshin SDK can be used from the Java server of \ac{glsp}. \cite{glsp-doc} With these functionalities, \ac{glsp} fits to create an easy accessible, intuitive application to create and apply Henshin model transformations, called Henshin Web.


The goal of this scientific work is to provide relevant information about the used technologies. Also existing web-based model transformation tools will be compared in the related work section. In section \ref{sec:deployment}, the deployment and usage of the Henshin Web editor will be discussed. The goal is to provide a web-based editor that can be used without any dependencies, like an installed \acs{ide} or other tools. The editor should be easy to access and use, so that it can be used by new users without any prior knowledge of model transformations or Henshin.

\section{Problem Statement}
\label{subsec:problem-statement}

\section{Research Questions}
\label{subsec:research-questions}

\section{Scope and Limitations}
\label{subsec:scope-limitations}

\section{Structure of the Thesis}
\label{subsec:structure-thesis}

The rest of the thesis is structured like this: In chapter \ref{sec:theoretical-background}, the theoretical background of the used technologies are presented. That includes the Eclipse Foundation, \ac{emf}, Henshin and \ac{glsp}. In chapter \ref{sec:related-work}, literature about model transformation software is sumerized and existing web-based model transformation tools are presented and compared. The requirements for the Henshin Web editor are discussed in chapter \ref{sec:requirements}. It describes potential users, the system scope and context as well as the functional and non functional requirements. The architecture of the Henshin Web editor is presented in chapter \ref{sec:architecture}. Deployment options are discussed in chapter \ref{sec:deployment}. Chapter \ref{sec:implementation} presents implementation details of core components of Henshin Web. The testing strategy of the application is discussed in chapter \ref{sec:testing}. Finally, the discussion and conclusion of the thesis is presented in chapter \ref{sec:discussion-conclusion}.