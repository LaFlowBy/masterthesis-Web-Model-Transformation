  \chapter{Related Work}
  \label{sec:related-work}

  \section{Scientific Literature}
  \label{subsec:related-scientific-literature}

  This section reviews relevant scientific literature that directly relates to the development of web-based modeling tools using \ac{glsp} and the migration of Eclipse-based modeling frameworks to web environments. The selected works provide practical insights into building \ac{glsp}-based applications, migrating \ac{mde} tools from Eclipse to web technologies, and improving the usability of model transformation frameworks like Henshin.

  \subsection{GLSP-Based Web Modeling Tools}

  \citeauthor{bork2023vision} present in \citeyear{bork2023vision} a comprehensive vision for flexible web-based modeling tools built with \ac{glsp} \cite{bork2023vision}. Their work addresses the same challenge that Henshin Web faces: creating modern diagram editors that work seamlessly across different platforms including web browsers, \ac{vscode}, Eclipse Theia, and the Eclipse desktop \ac{ide}. The authors describe how \ac{glsp}'s client-server architecture enables modeling tools to be platform-independent while maintaining professional-grade functionality. They emphasize that web-based modeling tools must provide the same quality as traditional desktop applications while leveraging web-specific advantages such as easier deployment, platform independence, and collaboration features. 
  
  The paper discusses architectural patterns for \ac{glsp}-based tools, including how to structure the server-side model management and client-side rendering. This directly informs the architecture of Henshin Web. The authors also address extensibility concerns, showing how \ac{glsp} applications can be customized through custom actions, tool palette items, and context menu contributions. These are features that Henshin Web utilizes for editing and applying transformation rules. Furthermore, the paper discusses integration possibilities with other Eclipse technologies, particularly \ac{emf}. This is crucial for Henshin Web's need to work with Ecore metamodels and \ac{xmi} instance files. The work validates the choice of \ac{glsp} as an appropriate framework for bringing Henshin to the web while maintaining compatibility with the existing Eclipse-based ecosystem.

  \subsection{Practical Experience with GLSP Development}

  \citeauthor{metin2023glsp} provide in \citeyear{metin2023glsp} valuable practical insights through their experience developing bigUML, a web-based \ac{uml} editor built with \ac{glsp} \cite{metin2023glsp}. Their paper reports on real-world challenges and solutions when building a production-ready \ac{glsp}-based modeling tool, making it highly relevant for the development of Henshin Web. The authors identify key architectural decisions, such as how to organize the server-side model state, handle concurrent editing scenarios, and implement complex editing operations that span multiple model elements. 
  
  They discuss performance considerations, particularly regarding the efficiency of transforming models into graphical representations when working with large diagrams. This concern is equally relevant when displaying complex Henshin transformation rules or large instance models. The paper also addresses the learning curve for developers new to \ac{glsp}, documenting common pitfalls and best practices that can speed up development. For example, they describe strategies for implementing custom validation logic, managing undo/redo operations, and integrating with external services. Their experience with deployment and operation of web-based modeling tools provides insights into hosting strategies, scalability considerations, and user access management. 
  
  The authors also discuss the benefits they observed in practice, including easier onboarding for new users who can access the tool immediately through a web browser without installation, and improved collaboration capabilities. These lessons learned directly inform the development approach for Henshin Web, helping to avoid common mistakes and adopt proven patterns for \ac{glsp}-based tool development.

  \subsection{Migration from Eclipse to Web Technologies}

  \citeauthor{domros2018moving} describes in \citeyear{domros2018moving} the migration of the KIELER modeling tool from the Eclipse desktop platform to web technologies using the Theia framework \cite{domros2018moving}. This work is particularly relevant as it addresses the same challenge that Henshin Web faces: adapting an existing Eclipse-based modeling tool for web environments. The thesis documents a systematic approach to migration, identifying which components can be reused, which require adaptation, and which need complete reimplementation. KIELER, like Henshin, is an Eclipse plugin that provides graphical modeling capabilities, making the migration challenges comparable. 
  
  Notably, while this thesis predates \ac{glsp}, KIELER uses Sprotty for diagram rendering—the same underlying \ac{svg}-based diagramming framework that \ac{glsp} is built upon. Later versions of KIELER integrated with \ac{glsp}, demonstrating the evolution path from Sprotty-based custom solutions to \ac{glsp}-based standardized approaches. The author analyzes the architectural differences between Eclipse and Theia, discussing how Eclipse's plugin system and extension points map to Theia's modular architecture. 
  
  Key findings include the importance of separating business logic from \ac{ui} concerns to enable cross-platform compatibility, and strategies for using the Language Server Protocol to share code between different frontend platforms. The thesis also evaluates the user experience improvements achieved through the web-based approach, including faster startup times, reduced installation complexity, and improved accessibility. 
  
  The author identifies challenges such as adapting keyboard shortcuts that conflict with browser behaviors, handling file system access in a sandboxed web environment, and managing the increased complexity of a client-server architecture. These insights provide valuable guidance for the Henshin Web migration strategy, particularly regarding which Eclipse-specific features to preserve and how to implement them in a web context.

  \subsection{Henshin Framework and Usability}

  \citeauthor{struber2017henshin} present in \citeyear{struber2017henshin} the current state of the Henshin framework, emphasizing its focus on usability for model transformation development \cite{struber2017henshin}. This work is essential context for Henshin Web as it documents the features and design philosophy of the framework being adapted for web environments. The authors describe how Henshin provides both graphical and textual syntax for defining transformation rules, supporting different user preferences and use cases. They discuss the algebraic graph transformation foundations that ensure formal correctness and verifiability of transformations. Henshin Web must preserve these foundations in its web-based implementation. 
  
  The paper presents Henshin's current tooling, including the Eclipse-based graphical editor for transformation rules, the tree-based editor for transformation units, and the interpreter for executing transformations. Understanding these existing tools is crucial for designing their web-based counterparts. The authors emphasize Henshin's integration with \ac{emf}, showing how transformation rules are typed over Ecore metamodels and how transformations operate on \ac{xmi} instance files. These are core concepts that Henshin Web must maintain. 
  
  The paper also discusses usability features such as visual differentiation between preserve, create, and delete actions through stereotypes, support for \acp{nac} and \acp{pac}, and parameterized rules. These features represent the baseline functionality that Henshin Web should provide to ensure users can perform the same transformation tasks as in the Eclipse version. The paper's focus on usability aligns directly with Henshin Web's goal of making model transformations more accessible by reducing the barrier to entry through a web-based interface.

  \section{Existing Tools and Technologies}
  \label{subsec:related-tools}

    There are many existing tools for model transformations. \citeauthor{kahani2019survey} created a survey in \citeyear{kahani2019survey} of various model transformation tools. They classified 60 different tools, including Henshin. In Figure \ref{fig:tools-environments}, you can see how many tools provide specific execution environments. 73\% of the tools provide plugins for the Eclipse \ac{ide}, and 20\% of the tools are integrated or dependent on other \acsp{ide}. 18\% have no \ac{ide} support, and only two tools are web-based. In total, 89\% of the tools have external dependencies such as an \ac{ide} or other tools. Dependencies often complicate the installation and usage of the tool. \cite{kahani2019survey}

  \begin{figure}[h]
    \centering
    \includegraphics[width=0.6\textwidth]{model-tools.png}
    \caption{Execution environments of model transformation tools. Image obtained from \cite{kahani2019survey}}
    \label{fig:tools-environments}
  \end{figure}

  One web-based tool included in the survey is \ac{atompm} \cite{atompm}. It is a web-based modeling tool to create \ac{dsml} environments, performing model transformations and manipulating and managing models. \cite{atompm} It was created in \citeyear{atompm} and supports all model transformations that are based on T-Core \cite{tcore}, a minimal common basis that allows interoperability between different model transformation languages. \cite{tcore} Metamodels can be defined with a simplified \ac{uml} language. The graphical modeling environment offers debugging and the ability to collaborate and share modeling artifacts in the browser. \cite{atompm}


  There are also other web-based tools for \ac{mde}. WebGME \cite{webGME} is a web-based modeling tool, created in \citeyear{webGME}. It allows to collaboratively design \acp{dsml} using model versioning and broadcasting changes to all active users. It supports prototypical inheritance, where any model can be instantiated recursively, so changes are propagated down the inheritance tree. It also provides scalability, collaborative modeling and model versioning. Metamodels and compositions can be created with WebGME, but no graph transformations can be applied to a model. Even though model transformations are not possible, the editor was one of  the first solutions for web-based modeling tools. \cite{webGME} The software provides extension points to customize or extend the software, but no model transformation capabilities were added by any available extension. \cite{webgme-website} The tool is still hosted and maintained, to be used for free. \cite{webgme-website}


  WebDPF \cite{webDPF} is another web-based modeling tool, published in \citeyear{webDPF}. Compared to WebGME and \ac{atompm}, it supports model navigation and element filter capabilities, a JavaScript editor for writing predicate semantics, reusability of transformation rules, partial model completion, and a termination analysis. These features try to improve the usability of the tool. \cite{webDPF} Even though the tool had improvments upon existing tools, the originally mentioned hosted WebDPF portal is offline by now. 


  There is also a \ac{glsp}-based Ecore metamodel editor, created by the \ac{glsp} development team. It was implemented with the \ac{glsp} version 0.9 but never updated further. It allows to create and edit \ac{emf} Ecore models in a Theia web editor. Even though the project cannot be used directly, due to the use of another source model format and breaking changes in major updates of the \ac{glsp} framework, it provides various classes that can be used as a template for the Henshin Web Ecore viewer. One example is the factory code that maps the \ac{emf} Ecore model to the graphical model. \cite{glsp-ecore-repo}
  The findings show, that there are many existing model transformation tools, but only very few web-based solutions, that provide an easy entry into \ac{mde} and model transformations. Henshin web tries to fill this gap.